\documentclass[10pt]{article}
\usepackage{amssymb,amsmath,times,url,graphicx,amsthm,alltt}
%\usepackage[pdftex,urlcolor=blue,pdfpagemode=none,pdfstartview=FitH]{hyperref}
\usepackage{my_packages}
\usepackage{tikz_packages}
\usepackage{wasysym}
%% url smaller font.
\makeatletter
\def\url@leostyle{%
  \@ifundefined{selectfont}{\def\UrlFont{\sf}}{\def\UrlFont{\small\ttfamily}}}
\makeatother
\urlstyle{leo}

%\usepackage[all,import]{xy}

\renewcommand{\baselinestretch}{1.2}
\date{}

\renewcommand{\thesubsection}{\arabic{subsection}. }
\renewcommand{\thesubsubsection}{\arabic{subsection}.\arabic{subsubsection} }

\theoremstyle{definition}
\newtheorem{prob}{Problem}[section]
%\renewcommand{\theprob}{\arabic{section}.\arabic{prob}}
\renewcommand{\theprob}{\arabic{prob}}

\newenvironment{subprob}%
{\renewcommand{\theenumi}{\alph{enumi}}\renewcommand{\labelenumi}{(\theenumi)}\begin{enumerate}}%
{\end{enumerate}}%

\newenvironment{matlab}
{\begin{alltt}\small\renewcommand{\baselinestretch}{1.2}\selectfont}%
{\end{alltt}}


\begin{document}



\setcounter{page}{1}
\pagestyle{plain}
\section*{MAE3145: Homework 5}
\vspace*{-0.4cm}
\noindent{Due date: \SI{2458078.2395}{\julianday} }%\\%\vspace*{0.5cm}

\begin{prob}
    Observations at a certain instant indicate the following data for an Earth satellite:
    \begin{align*}
        r_c = 4 R_{\oplus} \quad v_c = \SI{4.54}{\kilo\meter\per\second} \quad \gamma_c = \SI{-40}{\degree}
    \end{align*}
    
    \begin{subprob}
    \item Determine the following orbital characteristics of the original  orbit:
        \begin{align*}
        a \quad e \quad \mathbb{P} \quad \mathcal{E} \quad r_p \quad r_a \quad \nu_c \quad E_c \quad \parenth{t_c - T}
        \end{align*}
    \item In exactly \SI{8.5}{\hour}, a maneuver will be implemented.
        What are the orbital characteristics in \SI{8.5}{\hour} at the maneuver point? 
        In other words, find the orbital properties, \( r_m, v_m^{-}, \gamma_m^{-} \) immediately before applying the maneuver.
    \item The maneuver is defined as \( \norm{\Delta v} = \SI{1200}{\meter\per\second} \) (a very large maneuver) directed such that \( \alpha = \SI{30}{\degree} \) with respect to the original velocity vector at the maneuver point.
        Determine the following properties immediately after the impulsive maneuver:
        \begin{align*}
            a^{+} \quad e^{+} \quad \mathbb{P}^{+} \quad \mathcal{E}^{+} \quad r_p^{+} \quad r_a^{+} \quad \nu_c^{+} \quad E_c^{+} \quad \parenth{t_c - T}^{+} \quad \Delta \omega
        \end{align*}
        Ensure you draw a proper vector diagram!
    \item Create a plot with both the old and new orbit. 
        Mark the appropriate quantities on the plot.
    \end{subprob}
\end{prob}

\begin{prob}
    A vehicle has been successfully launched into an orbit such that \( e = 0.5 \) and \( a = 6.0 R_{\oplus} \). 
    Currently at \( t = t_0 \) the vehicle is located at perigee.
    A single in-plane maneuver is employed to circularize the orbit at geosynchronous \textbf{altitude} \( r = 6.6 R_{\oplus} \).
    \begin{subprob}
    \item Determine \( \vec r_1^{-} , \vec v_1^{-}, \gamma_1^{-} \)  at the maneuver point.
        These are the conditions on the orbit prior to the maneuver.
        The maneuver occurs at what value of \( \nu_1^{-} \)?
        What is the ``wait time'' until the maneuver, i.e. how long to go from perigee to the required maneuver point?
    \item Compute the required maneuver \( \norm{\Delta \vec v}, \alpha \).
        Ensure you include a proper vector diagram.
        What are the conditions on the orbit immediately following the maneuver, i.e. find \( \vec r_1^{+} , \vec v_1^{+}, \gamma_1^{+} \)?
    \item Plot the old and new orbits (together).
        On the plot, mark the following:
        \begin{align*}
            \vec r_0 \quad \vec r_1 \quad \vec v_1^{-} \quad \nu_1^{-} \quad \text{local horizon} \quad \gamma_1^{1} \quad \vec v_1^{+} \quad \gamma_1^{+} \quad \Delta \vec v \quad \alpha 
        \end{align*}
    \end{subprob}
\end{prob}

\begin{prob}
    A vehicle is currently in Earth orbit such that \( e = 0.75\) and \( a = 4.5 R_{\oplus} \).
    A single in-plane maneuver will be used to raise perigee and  lower apogee.
    New values are specified as \( r_p = 2.0 R_{\oplus}\) and  \( r_a = 6.0 R_{\oplus} \). 
    It is also required that perigee advance by \SI{35}{\degree}, i.e. \( \Delta \omega = + \SI{35}{\degree} \).

    \begin{subprob}
    \item At what location \( \nu \) in the original orbit should the maneuver be implemented?
        Determine \( \vec r_1^-, \vec v_1^{-}, \gamma_1^{-} \) at the maneuver point.
    \item Determine the maneuver (\( \norm{\Delta \vec v}, \alpha, \beta \)) to accomplish the objective.
        If there are two possibilities, \textbf{ALWAYS} choose the one with lowest cost. 
        Do not forget the vector diagrams!
        Can you deduce the lowest cost option  from the vector diagrams?
        What are the values of \( \vec v_1^{+}, \gamma^{+} \)?
    \item Define the maneuver in the VNC reference frame, i.e. write down the vector \( \Delta \vec v = a \hat v + b \hat n + c \hat c\).
        Create a plot with the old and new orbit as well as the properties: 
        \begin{align*}
            \vec r_0 \quad \vec r_1 \quad \vec v_1^{-} \quad \nu_1^{-} \quad \text{local horizon} \quad \gamma_1^{+} \quad \vec v_1^{+} \quad \gamma_1^{+} \quad \Delta \vec v \quad \alpha 
        \end{align*}
    \item Determine the position and velocity at \( \nu = \SI{250}{\degree} \) in the new orbit and mark this location on the plot.
        Determine the amount of time to go from the maneuver to \( \nu = \SI{250}{\degree} \) in the new orbit.
    \end{subprob}
\end{prob}

\begin{prob}
    As part of of an interplanetary mission, a spacecraft is in the following orbit around Mars (relative to a Mars-centered inertial coordinate frame):
    \begin{align*}
        a = 5 R_{\mars} \quad e = 0.5 \quad i = \SI{30}{\degree} \quad \Omega = \SI{45}{\degree} \quad \omega = - \SI{60}{\degree}
    \end{align*}

    At \( \nu = \SI{120}{\degree} \), the following maneuver is implemented
    \begin{align*}
        \Delta \vec v = 0.1 \hat x - 0.25 \hat y + 0.2 \hat z \si{\kilo\meter\per\second}.
    \end{align*}

    \begin{subprob}
    \item Transform the \( \Delta \vec v\) into the \( \hat r, \hat \theta, \hat h\) components corresponding to the original orbit.
        How much of the \( \Delta \vec v \) is out of the plane?
        Define this out of plane component as \( \Delta v_h \) and find the percentage as compared to the total \( \Delta \vec v \).
    \item Define \( \Delta \vec v_{r\theta} \) as the projection of \( \Delta \vec v \) in the orbital plane. 
        Determine \( \norm{\Delta \vec v_{r\theta}}, \beta, \phi \).
        Determine  \( \alpha \)  between the velocity vector in the original orbit and \( \Delta \vec v_{r\theta}\).
        Sketch this in-plane projection of components of the vector diagram.
        Add the unit vectors \( \hat r, \hat \theta\) and \( \hat v, \hat c \) to the sketch.
    \item To apply the maneuver, all position and velocities must be written in the same set of unit vectors, such as the inertial unit vectors \( \hat x, \hat y, \hat z \).
        Determine the new \( \vec r^{+}, \vec v^{+} \) immediately after the maneuver.
    \item Determine the orbital elements of the new orbit.
        Compare \( \hat r^- , \hat \theta^-, \hat h ^-\) (pre-maneuver in the original orbit) and \( \hat r^+, \hat \theta^+, \hat h^+\) (post-maneuver in the new orbit).
    \end{subprob}
\end{prob}

\begin{prob}
    In class, the optimality of the Hohmann transfer was discussed.
    Investigate it further by considering a  transfer between circular, coplanar Earth orbits.
    The initial and final orbital radii are \( r_1 = 1.25 R_{\oplus} \) and \( r_2 = 6.6 R_{\oplus} \), respectively.
\end{prob}

\begin{subprob}
\item For the Hohmann transfer, determine \( \norm{\Delta \vec v}, \alpha \) for each manuever, the total \( \norm{\Delta \vec v}\), and the time of flight. 
    Remember to draw a vector diagram!
\item Assume that this is a rendezvous problem, i.e. the vehicle should arrive at the final circular orbit to meet another spacecraft.
    What is the phase angle that is required between the two vehicles at departure from the lower orbit?
    Mark the pahse angle in a sketch of the transfer problem.
    What is the time before the same angular seperation is repeated?
\end{subprob}
\end{document}


