\documentclass[10pt]{article}
\usepackage{amssymb,amsmath,times,url,graphicx,amsthm,alltt}
%\usepackage[pdftex,urlcolor=blue,pdfpagemode=none,pdfstartview=FitH]{hyperref}
\usepackage{my_packages}
\usepackage{tikz_packages}
%% url smaller font.
\makeatletter
\def\url@leostyle{%
  \@ifundefined{selectfont}{\def\UrlFont{\sf}}{\def\UrlFont{\small\ttfamily}}}
\makeatother
\urlstyle{leo}

%\usepackage[all,import]{xy}

\renewcommand{\baselinestretch}{1.2}
\date{}

\renewcommand{\thesubsection}{\arabic{subsection}. }
\renewcommand{\thesubsubsection}{\arabic{subsection}.\arabic{subsubsection} }

\theoremstyle{definition}
\newtheorem{prob}{Problem}[section]
%\renewcommand{\theprob}{\arabic{section}.\arabic{prob}}
\renewcommand{\theprob}{\arabic{prob}}

\newenvironment{subprob}%
{\renewcommand{\theenumi}{\alph{enumi}}\renewcommand{\labelenumi}{(\theenumi)}\begin{enumerate}}%
{\end{enumerate}}%

\newenvironment{matlab}
{\begin{alltt}\small\renewcommand{\baselinestretch}{1.2}\selectfont}%
{\end{alltt}}


\begin{document}



\setcounter{page}{1}
\pagestyle{plain}
\section*{MAE3145: Homework 5}
\vspace*{-0.4cm}
\noindent{Due date: \SI{2458064.197916}{\julianday} }%\\%\vspace*{0.5cm}

\begin{prob}
    Observations at a certain instant indicate the following data for an Earth satellite:
    \begin{align*}
        r_c = 4 R_{\oplus} \quad v_c = \SI{4.54}{\kilo\meter\per\second} \quad \gamma_c = \SI{-40}{\degree}
    \end{align*}
    
    \begin{subprob}
    \item Determine the following orbital characteristics of the original  orbit:
        \begin{align*}
        a \quad e \quad \mathbb{P} \quad \mathcal{E} \quad r_p \quad r_a \quad \nu_c \quad E_c \quad \parenth{t_c - T}
        \end{align*}
    \item In exactly \SI{8.5}{\hour}, a manuever will be implemented.
        What are the orbital characteristics in \SI{8.5}{\hour} at the manuever point. 
        In other words, find the orbital properties, \( r_m, v_m^{-}, \gamma_m^{-} \) immediately before applying the maneuver.
    \item The manuever is defined as \( \norm{\Delta v} = \SI{1200}{\meter\per\second} \) ( a very large manuever) directed such that \( \alpha = \SI{30}{\degree} \) with respect to the original velocity vector at the maneuver point.
        Determine the following properties immediately after the impulsive maneuver:
        \begin{align*}
            a^{+} \quad e^{+} \quad \mathbb{P}^{+} \quad \mathcal{E}^{+} \quad r_p^{+} \quad r_a^{+} \quad \nu_c^{+} \quad E_c^{+} \quad \parenth{t_c - T}^{+} \quad \Delta \omega
        \end{align*}
        Ensure you draw a proper vector diagram!
    \item Create a plot with both the old and new orbit. 
        Mark the appropriate quantities on the plot.

    \end{subprob}
\end{prob}

\begin{prob}
    A vehicle has been successfully launched into an orbit such that \( e = 0.5 \) and \( a = 6.0 R_{\oplus} \). 
    Currently at \( t = t_0 \) the vehicle is located at perigee.
    A single in-plane maneuver is employed to circularize the orbit at geosynchronous altitude \( r = 6.6 R_{\oplus} \).
    \begin{subprob}
    \item Determine \( \vec r_1^{-} , \vec v_1^{-}, \gamma_1^{-} \)  at the maneuver point.
        These are the conditions on the orbit prior to the manuever.
        The manuever occurs at what value of \( \nu_1^{-} \)?
        What is the ``wait time'' until the manuever, i.e. how long to go from perigee to the required manuever point?
    \item Compute the required maneuver ( \( \norm{\Delta \vec v}, \alpha \).
        Ensure you include a proper vector diagram.
        What are the conditions on the orbit immediately following the maneuver, i.e. find \( \vec r_1^{+} , \vec v_1^{+}, \gamma_1^{+} \)?
    \item Plot the old and new orbits (together).
        On the plot, mark the following:
        \begin{align*}
            \vec r_0 \quad \vec r_1 \quad \vec v_1^{-} \quad \nu_1^{-} \quad \text{local horizon} \quad \gamma_1^{1} \quad \vec v_1^{+} \quad \gamma_1^{+} \quad \Delta \vec v \quad \alpha 
        \end{align*}
    \end{subprob}
\end{prob}

\begin{prob}
    A vehicle is currently in Earth orbit such that \( e = 0.75\) and \( a = 4.5 R_{\oplus} \).
    A single in-plane maneuver will be used to raise perigee and  lower apogee.
    New values are specified as \( r_p = 2.0 R_{\oplus}\) and  \( r_a = 6.0 R_{\oplus} \). 
    It is also required that perigee advance by \SI{35}{\degree}, i.e. \( \Delta \omega = + \SI{35}{\degree} \).

    \begin{subprob}
    \item At what location \( \nu \) in the original orbit should the maneuver be implemented?
        Determine \( \vec r_1, \vec v_1^{-}, \gamma_1^{-} \) at the manuever point.
    \item Determine the manuever ( \( \norm{\Delta \vec v}, \alpha, \beta \)) to accomplish the objective.
        If there are two possibilities, \textbf{ALWAYS} choose the one with lowest cost. 
        Do not forget the vector diagrams!
        Can you deduce the lowest cost option  from the vector diagrams?
        What are the values of \( \vec v_1^{+}, \gamma^{+} \)?
    \item Define the manuever in the VNC reference frame, i.e. write down the vector \( \Delta \vec v = a \hat v + b \hat n + c \hat c\).
        Create a plot with the old and new orbit as well as the properties: 
        \begin{align*}
            \vec r_0 \quad \vec r_1 \quad \vec v_1^{-} \quad \nu_1^{-} \quad \text{local horizon} \quad \gamma_1^{1} \quad \vec v_1^{+} \quad \gamma_1^{+} \quad \Delta \vec v \quad \alpha 
        \end{align*}
    \item Determine the position and velocity at \( \nu = \SI{250}{\degree} \) in the new orbit and mark this location on the plot.
        Determine the amount of time to go from the maneuver to \( \nu = \SI{250}{\degree} \) in the new orbit.
    \end{subprob}
\end{prob}
\end{document}


