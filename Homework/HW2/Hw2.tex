\documentclass[10pt]{article}
\usepackage[letterpaper,text={6.5in,8.7in},centering]{geometry}
\usepackage{amssymb,amsmath,times,url,subfigure,graphicx,theorem,alltt}
%\usepackage[pdftex,urlcolor=blue,pdfpagemode=none,pdfstartview=FitH]{hyperref}

%% url smaller font.
\makeatletter
\def\url@leostyle{%
  \@ifundefined{selectfont}{\def\UrlFont{\sf}}{\def\UrlFont{\small\ttfamily}}}
\makeatother
\urlstyle{leo}

%\usepackage[all,import]{xy}

\newcommand{\norm}[1]{\ensuremath{\left\| #1 \right\|}}
\newcommand{\abs}[1]{\ensuremath{\left| #1 \right|}}
\newcommand{\bracket}[1]{\ensuremath{\left[ #1 \right]}}
\newcommand{\braces}[1]{\ensuremath{\left\{ #1 \right\}}}
\newcommand{\parenth}[1]{\ensuremath{\left( #1 \right)}}
\newcommand{\ip}[1]{\ensuremath{\langle #1 \rangle}}
\newcommand{\refeqn}[1]{(\ref{eqn:#1})}
\newcommand{\reffig}[1]{Fig. \ref{fig:#1}}
\newcommand{\tr}[1]{\mbox{tr}\ensuremath{\negthickspace\bracket{#1}}}
\newcommand{\deriv}[2]{\ensuremath{\frac{\partial #1}{\partial #2}}}
\newcommand{\SO}{\ensuremath{\mathrm{SO(3)}}}
\newcommand{\T}{\ensuremath{\mathrm{T}}}
\newcommand{\so}{\ensuremath{\mathfrak{so}(3)}}
\newcommand{\SE}{\ensuremath{\mathrm{SE(3)}}}
\newcommand{\se}{\ensuremath{\mathfrak{se}(3)}}
\renewcommand{\Re}{\ensuremath{\mathbb{R}}}
\renewcommand{\S}{\ensuremath{\mathbb{S}}}
\newcommand{\aSE}[2]{\ensuremath{\begin{bmatrix}#1&#2\\0&1\end{bmatrix}}}
\newcommand{\ase}[2]{\ensuremath{\begin{bmatrix}#1&#2\\0&0\end{bmatrix}}}
\newcommand{\D}{\ensuremath{\mathbf{D}}}
\newcommand{\pair}[1]{\ensuremath{\left\langle #1 \right\rangle}}
\newcommand{\met}[1]{\ensuremath{\langle\!\langle #1 \rangle\!\rangle}}
\newcommand{\Ad}{\ensuremath{\mathrm{Ad}}}
\newcommand{\ad}{\ensuremath{\mathrm{ad}}}
\newcommand{\g}{\ensuremath{\mathfrak{g}}}

\renewcommand{\baselinestretch}{1.2}
\date{}

\renewcommand{\thesubsection}{\arabic{subsection}. }
\renewcommand{\thesubsubsection}{\arabic{subsection}.\arabic{subsubsection} }

\theoremstyle{plain}\theorembodyfont{\normalfont}
\newtheorem{prob}{Problem}[section]
%\renewcommand{\theprob}{\arabic{section}.\arabic{prob}}
\renewcommand{\theprob}{\arabic{prob}}

\newenvironment{subprob}%
{\renewcommand{\theenumi}{\alph{enumi}}\renewcommand{\labelenumi}{(\theenumi)}\begin{enumerate}}%
{\end{enumerate}}%


\begin{document}



\setcounter{page}{1}
\pagestyle{plain}
\section*{MAE3145: Homework 2}
\vspace*{-0.4cm}
\noindent{Due date: October 5, 2016}%\\%\vspace*{0.5cm}

%\begin{prob}
%Consider a spacecraft moving in a potential field specified as $U(\norm{\vec r})$. The equation of motion is given by
%\begin{align}
%\ddot{\vec r} = -\deriv{U}{\vec r}.
%\end{align}
%\begin{subprob}
%\item Show that the specific angular momentum $\vec h = \vec r \times \dot {\vec r}$ is still preserved.
%\item Show that the energy $\mathcal{E}=\frac{1}{2}\dot{\vec r}\cdot \dot{\vec r} + U(\norm{\vec r})$ is preserved.
%\end{subprob}
%\end{prob}

\begin{prob}
The relative motion of the two-body problem is described by
\begin{align}
\ddot{\vec r} = -\frac{\mu}{r^3}\vec r.
\end{align}
The specific angular momentum $\vec h$ and the eccentricity $\vec e$ are defined as follows:
\begin{align*}
\vec h = \vec r \times \vec v,\quad \vec e = \frac{\vec v\times\vec h}{\mu}- \frac{\vec r}{r}.
\end{align*}
In class, we found that $\vec h$ is fixed, i.e. $\dot{\vec h}=0$. Here, we wish to show $\vec e$ is fixed according to the following steps:
\begin{subprob}
\item Using (1), show that $\dfrac{d}{dt}(\vec v\times\vec h) = -\dfrac{\mu}{r^3} \vec r\times \vec h$.
\item Using the definition of $\vec h$, show that $\dfrac{1}{r^3}\vec r \times \vec h = \dfrac{\vec r \dot r - \dot{\vec r} r}{r^2}$.\\
(Hint: $\vec a\times (\vec b \times \vec c) = (\vec a \cdot \vec c)\vec b - (\vec a \cdot \vec b)\vec c$,\; $\vec r \cdot \vec r = r^2$, and $\vec r \cdot \dot{\vec r} = r\dot r$).
\item Show that $\dfrac{d}{dt}\dfrac{\vec r}{r} = -\dfrac{\vec r \dot r - \dot{\vec r} r}{r^2}$.
\item By combining the results of parts (a), (b), and (c), show that $\dfrac{d}{dt}\vec e=0$, i.e, the eccentricity vector is fixed.
\end{subprob}
\end{prob}

\begin{prob}
A satellite is on an elliptic orbit around the Earth with a perigee radius of $r_p=7000\,\mathrm{km}$ and an apogee radius of $r_a=70000\,\mathrm{km}$. Assume that the gravitational parameter and the radius of the Earth are $\mu=398600\,\mathrm{km^3/s^2}$, and $R_E=6378\,\mathrm{km}$, respectively. Determine the following parameters (specify units in $\mathrm{km},\,\mathrm{sec},\,\mathrm{degree}$). 
\begin{subprob}
\item eccentricity $e$
\item period $T$ 
\item specific energy $\mathcal{E}$
\item true anomaly $\theta$ at which the altitude is $1000\,\mathrm{km}$.
\item velocity $v_r,v_\theta$ at the point found in part (d).
\end{subprob}
\end{prob}

%\begin{prob}
%Consider an elliptic orbit with the distance to the periapsis $r_p$ and the distance to the apoapsis $r_a$. Using the properties summarized at the third page, show that the velocity at the periapsis and the velocity at the apoapsis are given by
%\begin{align*}
%v_p = \sqrt{\frac{2\mu}{r_a+r_p}\frac{r_a}{r_p}}
%,\quad
%v_a = \sqrt{\frac{2\mu}{r_a+r_p}\frac{r_p}{r_a}}.
%\end{align*}
%
%\end{prob}

\clearpage\newpage
\begin{prob}
The specific energy and angular momentum of several asteroids heading toward the Earth have been measured as follows:

\begin{center}
\begin{tabular}{c|c|c}\hline
Asteroid & $\mathcal{E}$ $(\mathrm{km^2/s^2})$ & $h$ $(\mathrm{km^2/s})$\\ \hline
1 & $1$ & $1\times 10^5$\\
2 & $100$ & $1\times 10^5$\\
3 & $0$ & $7\times 10^4$\\
4 & $0$ & $8\times 10^4$\\
5 & $10$ & $8\times 10^4$\\\hline
\end{tabular}
\end{center}
%
We wish to determine whether any asteroid is likely to hit the Earth. The trajectory of an asteroid is assumed to be the solution of the two-body problem of the asteroid and the Earth, where $\mu_E=398600\,\mathrm{km^3/s^2}$.

\begin{subprob}
\item Using the fact that $\mathcal{E}$ and $h$ are conserved, show that the distance at the periapsis $r_p$ satisfies the following quadratic equation:
\begin{gather}
2\mathcal{E}\, r_p^2 + 2\mu_E\, r_p -h^2=0. 
\end{gather}
(Hint: at the periapsis, $h=rv$ since $\vec r$ is perpendicular to $\vec v$.)
\item Calculate $r_p$ for all asteroids, and determine which asteroid will hit the surface of the Earth: an asteroid will hit the Earth if $r_p < R_E=6378\,\mathrm{km}$.\\
(Hint: In Matlab, the quadratic equation $ax^2+bx+c=0$ can be solved by the command \texttt{roots([a b c])}.)
\item For each asteroid that hits the surface of the Earth, calculate its impact velocity at the surface of the Earth.\\
(Note: the impact velocity is not same as the velocity at the periapsis.)
\item For each asteroid that does not hit the surface of the Earth, calculate its velocity when it is closest to the Earth.
\end{subprob}
\end{prob}

\end{document}


\begin{prob}
Consider Asteroid 5 discussed above. Its specific energy and specific angular momentum are given by $\mathcal{E}=10\,\mathrm{km^2/s^3}$, and $h=8\times 10^4\,\mathrm{km^2/s}$. We want to determine the time after periapsis passage $t$ when the true anomaly is $\theta=100^\circ$.
\begin{subprob}
\item Compute the semi-major axis $a$, and the eccentricity $e$.
\item Compute the maximum true anomaly $\theta_\infty$. Is $\theta < \theta_\infty$?
\item Compute the hyperbolic eccentric anomaly $F$, and the hyperbolic mean anomaly $M_e$.
\item Show that the time after the periapsis passage is given by $t=0.6979\,\mathrm{hrs}$.
\end{subprob}
\end{prob}

\begin{prob}
An Earth-orbiting satellite has a period of $T=15.743$ hours and a periapsis radius $r_p=12756\,\mathrm{km}$. We want to determine the location of this satellite at time $t=1$ hour after periapsis passage.
\begin{subprob}
\item Compute the semi-major axis $a$, and the eccentricity $e$.
\item Compute the mean anomaly $M_e$.
\item Write a Matlab program to compute the eccentric anomaly $E$.
\item Show that the true anomaly is given by $\theta=84.2850^\circ$.
\end{subprob}
(Hint: if you want to verify your code, check that your code gives $\theta=\pi$ when $t=T/2$.)
\end{prob}


\end{document}


%\begin{prob}
%For the Asteroid 5 in Problem 2, we want to determine the time after periapsis passage $t$ when the true anomaly is $\theta=100^\circ$.
%\begin{subprob}
%\item Compute the semi-major axis $a$, and the eccentricity $e$.
%\item Compute the maximum true anomaly $\theta_\infty$. Is $\theta < \theta_\infty$?
%\item Compute the hyperbolic eccentric anomaly $F$, and the hyperbolic mean anomaly $M_e$.
%\item Compute the time after the periapsis passage $t$.
%\end{subprob}
%\end{prob}
%
%\clearpage\newpage
%
%\begin{prob}
%An Earth-orbiting satellite has a period of $T=15.743$ hours and a periapsis radius $r_p=12756\,\mathrm{km}$. We want to determine the location of this satellite at time $t=1$ hour after periapsis passage.
%\begin{subprob}
%\item Compute the semi-major axis $a$, and the eccentricity $e$.
%\item Compute the mean anomaly $M_e$.
%\item Write a Matlab program to compute the eccentric anomaly $E$.
%\item Compute the true anomaly $\theta$.
%\end{subprob}
%(Hint: if you want to verify your code, check that your code gives $\theta=\pi$ when $t=T/2$.)
%\end{prob}

\clearpage\newpage
\section*{Summary of Orbital Properties}
\pagestyle{empty}
\subsection*{Any Type of Orbit}

\begin{alignat*}{3}
\vec h & = \vec r \times \vec v,&\qquad 
\vec r & = r \hat u_r&\qquad
\vec v & = v_r \hat u_r + v_\theta \hat u_\theta,\\
h & = rv\cos\gamma = rv_\theta = r^2\dot\theta, &
r & = \frac{h^2/\mu}{1+e \cos\theta}, &
v_r & = \frac{\mu}{h}e\sin\theta = \dot r,\\
\vec e & = \frac{\vec v\times\vec h}{\mu}- \frac{\vec r}{r}, &
r_p & = \frac{h^2/\mu}{1+e}, &
v_\theta & = \frac{\mu}{h}(1+e\cos\theta) = r\dot\theta,\\
&&
\mathcal{E} & = \frac{1}{2} v^2 - \frac{\mu}{r} = -\frac{1}{2}\frac{\mu^2}{h^2}(1-e^2),&
\gamma & = \frac{v_r}{v_\theta} = \frac{e\sin\theta}{1+e\cos\theta}.
\end{alignat*}

\subsection*{Circular Orbits: ($e=0$)}

\begin{alignat*}{3}
v & = \sqrt{\frac{\mu}{r}},&\qquad
\mathcal{E} & = -\frac{\mu}{2r},&\qquad
T & = \frac{2\pi}{\sqrt{\mu}} r^{3/2}.
\end{alignat*}

\subsection*{Elliptic Orbits: ($0<e<1$)}

\begin{alignat*}{3}
r & = \frac{a(1-e^2)}{1+e\cos\theta},&\qquad
a & = \frac{h^2/\mu}{1-e^2} =\frac{1}{2}(r_a+r_p),&\qquad
T & = \frac{2\pi}{\sqrt{\mu}} a^{3/2},\\
r_p & = \frac{h^2/\mu}{1+e} = a(1-e),&
b & = a \sqrt{1-e^2},&
e & = \frac{r_a-r_p}{r_a+r_p},\\
r_a & = \frac{h^2/\mu}{1-e} = a (1+e),&
\mathcal{E} & = -\frac{\mu}{2a}.
\end{alignat*}

\subsection*{Parabolic Orbits: ($e=1$)}
\begin{alignat*}{2}
v & = \sqrt{\frac{2\mu}{r}},&\qquad
\mathcal{E} & = 0.
\end{alignat*}

\subsection*{Hyperbolic Orbits: ($e>1$)}

\begin{align*}
r & = \frac{a(e^2-1)}{1+e\cos\theta},&\qquad
a & = \frac{h^2/\mu}{e^2-1},&\qquad
\theta_\infty & = \cos^{-1} (-1/e),\\
r_p & = a(e-1),&
b & = a \sqrt{e^2-1},&
\beta & = \cos^{-1} (1/e),\\
&&
\mathcal{E} & = \frac{\mu}{2a}.
\end{align*}





\end{document}

