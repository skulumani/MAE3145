\documentclass[10pt]{article}
\usepackage[letterpaper,text={6.5in,8.7in},centering]{geometry}
\usepackage{amssymb,amsmath,times,url,subfigure,graphicx,theorem,alltt,eepic,epic,color,tikz}
%\usepackage[pdftex,urlcolor=blue,pdfpagemode=none,pdfstartview=FitH]{hyperref}

%% url smaller font.
\makeatletter
\def\url@leostyle{%
  \@ifundefined{selectfont}{\def\UrlFont{\sf}}{\def\UrlFont{\small\ttfamily}}}
\makeatother
\urlstyle{leo}

%\usepackage[all,import]{xy}

\newcommand{\norm}[1]{\ensuremath{\left\| #1 \right\|}}
\newcommand{\abs}[1]{\ensuremath{\left| #1 \right|}}
\newcommand{\bracket}[1]{\ensuremath{\left[ #1 \right]}}
\newcommand{\braces}[1]{\ensuremath{\left\{ #1 \right\}}}
\newcommand{\parenth}[1]{\ensuremath{\left( #1 \right)}}
\newcommand{\ip}[1]{\ensuremath{\langle #1 \rangle}}
\newcommand{\refeqn}[1]{(\ref{eqn:#1})}
\newcommand{\reffig}[1]{Fig. \ref{fig:#1}}
\newcommand{\tr}[1]{\mbox{tr}\ensuremath{\negthickspace\bracket{#1}}}
\newcommand{\deriv}[2]{\ensuremath{\frac{\partial #1}{\partial #2}}}
\newcommand{\SO}{\ensuremath{\mathrm{SO(3)}}}
\newcommand{\T}{\ensuremath{\mathrm{T}}}
\newcommand{\so}{\ensuremath{\mathfrak{so}(3)}}
\newcommand{\SE}{\ensuremath{\mathrm{SE(3)}}}
\newcommand{\se}{\ensuremath{\mathfrak{se}(3)}}
\renewcommand{\Re}{\ensuremath{\mathbb{R}}}
\renewcommand{\S}{\ensuremath{\mathbb{S}}}
\newcommand{\aSE}[2]{\ensuremath{\begin{bmatrix}#1&#2\\0&1\end{bmatrix}}}
\newcommand{\ase}[2]{\ensuremath{\begin{bmatrix}#1&#2\\0&0\end{bmatrix}}}
\newcommand{\D}{\ensuremath{\mathbf{D}}}
\newcommand{\pair}[1]{\ensuremath{\left\langle #1 \right\rangle}}
\newcommand{\met}[1]{\ensuremath{\langle\!\langle #1 \rangle\!\rangle}}
\newcommand{\Ad}{\ensuremath{\mathrm{Ad}}}
\newcommand{\ad}{\ensuremath{\mathrm{ad}}}
\newcommand{\g}{\ensuremath{\mathfrak{g}}}

\renewcommand{\baselinestretch}{1.2}
\date{}

\renewcommand{\thesubsection}{\arabic{subsection}. }
\renewcommand{\thesubsubsection}{\arabic{subsection}.\arabic{subsubsection} }

\theoremstyle{plain}\theorembodyfont{\normalfont}
\newtheorem{prob}{Problem}[section]
%\renewcommand{\theprob}{\arabic{section}.\arabic{prob}}
\renewcommand{\theprob}{\arabic{prob}}

\newenvironment{subprob}%
{\renewcommand{\theenumi}{\alph{enumi}}\renewcommand{\labelenumi}{(\theenumi)}\begin{enumerate}}%
{\end{enumerate}}%

\newenvironment{matlab}
{\begin{alltt}\small\renewcommand{\baselinestretch}{1.2}\selectfont}%
{\end{alltt}}

\newcommand*\circled[1]{%
  \tikz[baseline=(C.base)]\node[draw,circle,inner sep=0.5pt](C) {#1};\!
}

\begin{document}

\pagestyle{empty}
\section*{MAE3145: Solution for Homework 4, Question 3}
%\vspace*{-0.4cm}
%\noindent{Due date: November 19, 2012}%\\%\vspace*{0.5cm}


\paragraph{Code}
\begin{matlab}
clear all;
close all;

mu=398600;

rA=7000;
rB=14000;
vA1=12;

%% a

E1=1/2*vA1^2-mu/rA;
h1=vA1*rA;

e1=sqrt(2*E1*h1^2/mu^2+1);
a1=h1^2/mu/(e1^2-1);

vA2=sqrt(2*mu/(rA+rB)*rB/rA);
delVA=vA2-vA1

%% b

a2=1/2*(rA+rB);
T2=2*pi/sqrt(mu)*a2^(3/2);
t1=T2/2

%% c

T4=2*pi/sqrt(mu)*rB^(3/2);
thetaC=t1/T4*2*pi

t2=T4

%% d
T3=t2-t1
a3=(T3*sqrt(mu)/2/pi)^(2/3)
% check T3=2*pi/sqrt(mu)*a3^(3/2)

rC=2*a3-rB

%% e

vB3=sqrt(2*mu/(rB+rC)*rC/rB)
vB2=sqrt(2*mu/(rA+rB)*rA/rB)
delVB1=vB3-vB2

%% f

vB4=sqrt(mu/rB)
delVB2=vB4-vB3

%% g

delVT=abs(delVA)+abs(delVB1)+abs(delVB2)
\end{matlab}

\paragraph{Results}
\begin{matlab}
>> prob2
delVA =
   -3.2866
t1 =
   5.3538e+03
thetaC =
    2.0405
t2 =
   1.6486e+04
T3 =
   1.1132e+04
a3 =
   1.0775e+04
rC =
   7.5508e+03
vB3 =
    4.4667
vB2 =
    4.3567
delVB1 =
    0.1100
vB4 =
    5.3359
delVB2 =
    0.8692
delVT =
    4.2657
\end{matlab}

\end{document}

