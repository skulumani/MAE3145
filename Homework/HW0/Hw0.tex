\documentclass[10pt]{article}
% \usepackage[letterpaper,text={6.5in,8.7in},centering]{geometry}
\usepackage{amssymb,amsmath,times,url,graphicx,amsthm,alltt}
%\usepackage[pdftex,urlcolor=blue,pdfpagemode=none,pdfstartview=FitH]{hyperref}
\usepackage{my_packages}
\usepackage{tikz_packages}
%% url smaller font.
\makeatletter
\def\url@leostyle{%
  \@ifundefined{selectfont}{\def\UrlFont{\sf}}{\def\UrlFont{\small\ttfamily}}}
\makeatother
\urlstyle{leo}

%\usepackage[all,import]{xy}

\renewcommand{\baselinestretch}{1.2}
\date{}

\renewcommand{\thesubsection}{\arabic{subsection}. }
\renewcommand{\thesubsubsection}{\arabic{subsection}.\arabic{subsubsection} }

\theoremstyle{definition}
\newtheorem{prob}{Problem}[section]
%\renewcommand{\theprob}{\arabic{section}.\arabic{prob}}
\renewcommand{\theprob}{\arabic{prob}}

\newenvironment{subprob}%
{\renewcommand{\theenumi}{\alph{enumi}}\renewcommand{\labelenumi}{(\theenumi)}\begin{enumerate}}%
{\end{enumerate}}%


\begin{document}

\pagestyle{empty}
\section*{MAE3145: Homework 0}
\vspace*{-0.4cm}
\noindent{Due date: September 9, 2015}%\\%\vspace*{0.5cm}


\begin{prob}
Two vectors $\vec a,\vec b\in\Re^3$ are given as follows:
\begin{align*}
\vec a = [a_1,\,a_2,\, a_3]^T,\quad \vec b = [b_1,\,b_2,\, b_3]^T.
\end{align*}
The dot product of two vectors is a scalar defined as
\begin{align}
\vec a \cdot \vec b = a_1b_1+a_2b_2+a_3b_3.\label{eqn:dot1}
\end{align}
Let $a=\sqrt{\vec a \cdot \vec a}$,\; $b=\sqrt{\vec b \cdot \vec b}$\; be the length of the vectors $\vec a$, and $\vec b$, respectively. Let $\theta$ be the angle between $\vec a$ and $\vec b$. The dot product can be re-written in these variables as
\begin{align}
\vec a \cdot \vec b = a b \cos\theta
\end{align}

\begin{subprob}
\item Suppose that $\vec a\cdot\vec b =0$. What is the corresponding value of the angle $\theta$. (Assume that $\vec a\neq 0$, $\vec b\neq 0$, and $0\leq\theta < 2 \pi$.)
%\item Suppose that $\vec a \cdot \vec b < 0$. What is the possible range of the angle $\theta$. (Assume $0\leq\theta < 2 \pi$.)
\item Let $\vec a = [1,\,3,\,-2]^T$, $\vec b =[-4,\,-1,-2]^T$. Find $\vec a \cdot \vec b$ using \refeqn{dot1}.
\item For $\vec a$ and $\vec b$ defined at (c), find the angle between those two vectors.
\item Check that your answer of (d) is consistent with your answer to (b).
\end{subprob}


\end{prob}


\begin{prob}
The cross product of two vectors is another vector defined as
\begin{align}
\vec a \times \vec b = [a_2b_3-a_3b_2,\; a_3b_1-a_1b_3,\; a_1b_2-a_2b_1]^T.\label{eqn:cross1}
\end{align}
Alternatively, it is written as the determinant of the following matrix:
\begin{align}
\vec a \times \vec b = \det
\begin{bmatrix}
\hat i & \hat j & \hat k\\
a_1 & a_2 & a_3\\
b_1 & b_2 & b_3
\end{bmatrix}.\label{eqn:cross2}
\end{align}
We can show that its magnitude is given by
\begin{align}
\| \vec a \times \vec b \| = a b \sin\theta,\label{eqn:cross3}
\end{align}
and the direction of $\vec a \times \vec b$ is determined by the right-handed rule. Note that $\vec a\times \vec b$ is always perpendicular to either $\vec a$, $\vec b$, or the plane spanned by $\vec a$ and $\vec b$.

\begin{subprob}
\item Suppose that $\vec a \times \vec b = \vec 0$. What is the corresponding value of the angle $\theta$. (Assume $0\leq\theta < 2 \pi$.)
\end{subprob}
For the remaining parts (b)-(e), let $\vec a = [1,\,3,\,-2]^T$, $\vec b =[-4,\,-1,-2]^T$. 
\begin{subprob}\setcounter{enumi}{1}
\item Show that $\vec c = \vec a \times \vec b=[-8,\,10,\,11]^T$ using \refeqn{cross1} or \refeqn{cross2}. Also compute its length using $c = \sqrt{\vec c \cdot \vec c}$.
\item Compute the length $\| \vec a \times \vec b \|$ using \refeqn{cross3}, and show that it is equal to your second answer to (b).
\item Find the angle between $\vec a$ and $\vec c$. 
\end{subprob}
\end{prob}

\begin{prob}
Write the law of cosines relating each of the three sides and one interior angle of a triangle.
There should be three equations.
\begin{figure}[h]
    \centering
\begin{tikzpicture}
    [
    point/.style = {draw, circle, fill = black, inner sep = 1pt},
    dot/.style = {draw, circle, fill = black, inner sep = 0.2pt}
    ]
    \pgfmathsetlengthmacro{\sidelength}{4cm}
    \pgfmathsetlengthmacro{\arcradius}{0.8cm}
    \pgfmathsetlengthmacro{\arclabelradius}{\arcradius + 5pt}
    % three corners of the triangle
    \node (l) at (0, 0) [point, label = {}] {};
    \node (r) at (4, 0) [point, label = {}] {};
    \node (t) at (2, 3.464) [point, label = {}] {};

    \drawlabeledarcdelta{(l)}{0}{60}{\arcradius}{$\gamma$}{\arclabelradius};
    \drawlabeledarcdelta{(r)}{120}{60}{\arcradius}{$\alpha$}{\arclabelradius};
    \drawlabeledarcdelta{(t)}{240}{60}{\arcradius}{$\beta$}{\arclabelradius};

    \draw (l) -- node [below] {$B$} (r) -- node [right] {$C$} (t) -- node [left] {$A$} (l);
\end{tikzpicture}
\end{figure}
\begin{subprob}
\item Show that the law of cosines reduces to the well-known Ptyhagorean formula for a right triangle.
\end{subprob}
\end{prob}

\begin{prob}
    In the matrix formula, \( A x = b \), where \( b \in \R^{n\times1} \) is a column vector, what is are the dimensions of \( A \) and \(x \).
\end{prob}

\begin{prob}
    Consider the matrix 
    \begin{equation}
    A = \begin{bmatrix} a & b \\ c & d \end{bmatrix}.
    \end{equation}
    \begin{subprob}
    \item What is \( A^{-1} \)?
    \item What is \( A^{-1} A \)?
    \item What is \( A A^{-1} \)?
    \end{subprob}
\end{prob}

\begin{prob}
Convert the vector \( x = \begin{pmatrix} 6 & \frac{\pi}{2} & \pi \end{pmatrix}\) from spherical to rectangular coordinates.
The spherical coordinates are defined as \( x = \begin{pmatrix} r & \phi & \theta \end{pmatrix}\) with the angles given in radians.
    \begin{figure}[h]
        \centering
        \tdplotsetmaincoords{60}{110}
        \pgfmathsetmacro{\rvec}{1}
        \pgfmathsetmacro{\thetavec}{50}
        \pgfmathsetmacro{\phivec}{50}

        \begin{tikzpicture}[scale=3, tdplot_main_coords]
            \coordinate (O) at (0, 0, 0);
            \draw[thick, ->] (O) -- (1, 0, 0) node[anchor=north east] {$x$};
            \draw[thick, ->] (O) -- (0, 1, 0) node[anchor=north west] {$y$};
            \draw[thick, ->] (O) -- (0, 0, 1) node[anchor=south] {$z$};
            \tdplotsetcoord{P}{\rvec}{\phivec}{\thetavec}
            \draw[thick, ->, color=red] (O) -- (P) node[above right] {$r$};
            \draw[dashed, color=red] (O) -- (Pxy);
            \draw[dashed, color=red] (P) -- (Pxy);
            \tdplotdrawarc{(O)}{0.2}{0}{\thetavec}{anchor=north}{$\theta$};
            \tdplotsetthetaplanecoords{\thetavec}
            \tdplotdrawarc[tdplot_rotated_coords]{(0, 0, 0)}{0.2}{0}{\phivec}{anchor = south west}{$\phi$};
        \end{tikzpicture}
    \end{figure}
    \begin{subprob}
    \item Sketch the vector.
    \item What is the magnitude of the vector?
    \end{subprob}
\end{prob}

\begin{prob}
    Consider the function
    \begin{equation}
        y = \exp{x}.
    \end{equation}
    \begin{subprob}
    \item What is the Taylor Series approximation of \( y (x) \) about the point \( x = a \)?
    \item Using the approximation, find \( y (0.1) \) using a third order Taylor series about the point \( a = 0.0 \).
    \item Estimate the error of your answer and compare it to the true value of \( y(0.1)\).
    \end{subprob}
\end{prob}
\begin{prob}
The following vector identities will be used in this class later.
\begin{gather}
\vec x \cdot (\vec y \times \vec z) =
\vec y \cdot (\vec z \times \vec x) =
\vec z \cdot (\vec x \times \vec y),\label{eqn:iden1}\\
\vec x \times (\vec y \times \vec z) =
(\vec x \cdot \vec z) \vec y - (\vec x\cdot \vec y) \vec z.
\end{gather}
\begin{subprob}
\item Using \refeqn{iden1}, show that $\vec x$ is perpendicular to $\vec x\times \vec y$, i.e., show $\vec x\cdot (\vec x\times \vec y)=0$.
\item Suppose that $\vec q$ is a unit vector, i.e., $\|\vec q\|=1$. Show that $- \vec q \times (\vec q \times \vec x)$ is the orthogonal projection of $\vec x$ to the plane normal to $\hat q$, i.e., show that $- \vec q \times (\vec q \times \vec x)=\vec x - (\vec q\cdot \vec x) \vec q$.
\end{subprob}

\end{prob}


\begin{prob}
A satellite is on a circular orbit around the Earth, i.e. the trajectory of the satellite is a circle centered at the center of the Earth. We observe that the satellite is located at the following point from the center of the Earth:
\begin{align*}
\vec r_A = 3741.7\,\hat i +    5612.5\, \hat j  -1870.8\,\hat k\,(\mathrm{km}).
\end{align*}
Sometime later, we observe that the satellite moved to the following position:
\begin{align*}
\vec r_B = -441.8\,\hat i +    6627.5\, \hat j +2209.2\,\hat k\,(\mathrm{km}).
\end{align*}

\begin{subprob}
\item Find the radius of the circular orbit. (Specify the unit!)
\item What is the rotation angle of the satellite on its circular orbit, i.e. find the angle between $\vec r_A$ and $\vec r_B$.
\item Find a \textbf{unit} vector, namely $\hat h$ that is perpendicular to the orbital plane. (Hint: find any vector, say $\vec h$, that is perpendicular to both of $\vec r_A$ and $\vec r_B$, and normalize it with its length, i.e. $\hat h = \vec h / h$, where $h=\sqrt{\vec h \cdot \vec h}$).
\end{subprob}

\end{prob}


\begin{prob}
Email a casual photo of yourself to \url{tylee@gwu.edu} with your name. Please, make the email subject start with \texttt{MAE3145}.  This is NOT mandatory, but it would be very helpful to me memorize the name of the students. 
\end{prob}

%\begin{alltt}\footnotesize\renewcommand{\baselinestretch}{1.2}\selectfont
%clear all;
%close all;
%
%theta0=pi/4;
%dottheta0=0;
%X0=[theta0; dottheta0];
%
%[t,X]=ode45(@eomPend,[0 10],X0);
%
%theta=X(:,1);
%dottheta=X(:,2);
%
%plot(theta,dottheta);
%\end{alltt}
\end{document}

