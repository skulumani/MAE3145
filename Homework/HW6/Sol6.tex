\documentclass[10pt]{article}
\usepackage[letterpaper,text={6.5in,8.7in},centering]{geometry}
\usepackage{amssymb,amsmath,times,url,subfigure,graphicx,theorem,alltt,eepic,tikz}
%\usepackage[pdftex,urlcolor=blue,pdfpagemode=none,pdfstartview=FitH]{hyperref}

%% url smaller font.
\makeatletter
\def\url@leostyle{%
  \@ifundefined{selectfont}{\def\UrlFont{\sf}}{\def\UrlFont{\small\ttfamily}}}
\makeatother
\urlstyle{leo}

%\usepackage[all,import]{xy}

\newcommand{\norm}[1]{\ensuremath{\left\| #1 \right\|}}
\newcommand{\abs}[1]{\ensuremath{\left| #1 \right|}}
\newcommand{\bracket}[1]{\ensuremath{\left[ #1 \right]}}
\newcommand{\braces}[1]{\ensuremath{\left\{ #1 \right\}}}
\newcommand{\parenth}[1]{\ensuremath{\left( #1 \right)}}
\newcommand{\ip}[1]{\ensuremath{\langle #1 \rangle}}
\newcommand{\refeqn}[1]{(\ref{eqn:#1})}
\newcommand{\reffig}[1]{Fig. \ref{fig:#1}}
\newcommand{\tr}[1]{\mbox{tr}\ensuremath{\negthickspace\bracket{#1}}}
\newcommand{\deriv}[2]{\ensuremath{\frac{\partial #1}{\partial #2}}}
\newcommand{\SO}{\ensuremath{\mathrm{SO(3)}}}
\newcommand{\T}{\ensuremath{\mathrm{T}}}
\newcommand{\so}{\ensuremath{\mathfrak{so}(3)}}
\newcommand{\SE}{\ensuremath{\mathrm{SE(3)}}}
\newcommand{\se}{\ensuremath{\mathfrak{se}(3)}}
\renewcommand{\Re}{\ensuremath{\mathbb{R}}}
\renewcommand{\S}{\ensuremath{\mathbb{S}}}
\newcommand{\aSE}[2]{\ensuremath{\begin{bmatrix}#1&#2\\0&1\end{bmatrix}}}
\newcommand{\ase}[2]{\ensuremath{\begin{bmatrix}#1&#2\\0&0\end{bmatrix}}}
\newcommand{\D}{\ensuremath{\mathbf{D}}}
\newcommand{\pair}[1]{\ensuremath{\left\langle #1 \right\rangle}}
\newcommand{\met}[1]{\ensuremath{\langle\!\langle #1 \rangle\!\rangle}}
\newcommand{\Ad}{\ensuremath{\mathrm{Ad}}}
\newcommand{\ad}{\ensuremath{\mathrm{ad}}}
\newcommand{\g}{\ensuremath{\mathfrak{g}}}

\renewcommand{\baselinestretch}{1.2}
\date{}

\renewcommand{\thesubsection}{\arabic{subsection}. }
\renewcommand{\thesubsubsection}{\arabic{subsection}.\arabic{subsubsection} }

\theoremstyle{plain}\theorembodyfont{\normalfont}
\newtheorem{prob}{Problem}[section]
%\renewcommand{\theprob}{\arabic{section}.\arabic{prob}}
\renewcommand{\theprob}{\arabic{prob}}

\newenvironment{subprob}%
{\renewcommand{\theenumi}{\alph{enumi}}\renewcommand{\labelenumi}{(\theenumi)}\begin{enumerate}}%
{\end{enumerate}}%

\newenvironment{matlab}
{\begin{alltt}\small\renewcommand{\baselinestretch}{1.2}\selectfont}%
{\end{alltt}}

\newcommand*\circled[1]{%
  \tikz[baseline=(C.base)]\node[draw,circle,inner sep=0.5pt](C) {#1};\!
}

\begin{document}

\pagestyle{empty}
\section*{MAE3145: Solution 5}

Every quantity is expressed in the units of $\mathrm{km}$, $\mathrm{sec}$, $\mathrm{rad}$ if not specified.

\begin{prob}
\begin{matlab}
clear all;
close all;
mu=398600;

rA=7000;
thetaA=210*pi/180;
vA=sqrt(mu/rA);
h1=rA*vA;
e1=0;
rB=6378;
thetaB=0*pi/180;

% (a)
vA1_vec=mu/h1*[-sin(thetaA) (e1+cos(thetaA))]

% (b)
e2=(rB-rA)/(rA*cos(thetaA)-rB*cos(thetaB))
h2=sqrt(mu*rA*rB)*sqrt((cos(thetaA)-cos(thetaB))/(rA*cos(thetaA)-rB*cos(thetaB)))

% (c)
vA2_vec=mu/h2*[-sin(thetaA) (e2+cos(thetaA))]

% (d)
delvA_vec=vA2_vec-vA1_vec

% (e)
vB3_vec=mu/h2*[-sin(thetaB) (e2+cos(thetaB))]

>> prob1
vA1_vec =
    3.7730   -6.5351
e2 =
    0.0500
h2 =
   5.1666e+04
vA2_vec =
    3.8575   -6.2956
delvA_vec =
    0.0844    0.2395
vB3_vec =
         0    8.1007
\end{matlab}


\end{prob}

\newpage
\begin{prob}
\begin{matlab}
clear all;
close all;
mu=398600;

ra=13000;
rp=8000;

% (a) Position and Velocity at P
e1=(ra-rp)/(ra+rp);
h1=sqrt(rp*mu*(1+e1));

rP_vec=[rp 0]
vP1_vec=[0 h1/rp]

% (b) Position and Velocity at D

thetaD=pi/2;
rD=h1^2/mu/(1+e1*cos(thetaD));

rD_vec=rD*[cos(thetaD) sin(thetaD)]
vD1_vec=mu/h1*[-sin(thetaD) (e1+cos(thetaD))]

% (c) Time at D
ED=2*atan(sqrt((1-e1)/(1+e1))*tan(thetaD/2));
MeD=ED-e1*sin(ED);
a1=1/2*(ra+rp);
T1=2*pi/sqrt(mu)*a1^(3/2);
tD=MeD/2/pi*T1

% Time at C
thetaC=30*pi/180;
EC=2*atan(sqrt((1-e1)/(1+e1))*tan(thetaC/2));
MeC=EC-e1*sin(EC);
tC=MeC/2/pi*T1

tPD=tD-tC

% (d) Lambert Problem
[vP2_vec vD2_vec a e h]=LambertProb(rP_vec,rD_vec,tPD,mu)

% (e)
delvP=vP2_vec-vP1_vec
delvD=vD1_vec-vD2_vec

% (f)
delV=norm(delvP)+norm(delvD)

>> prob2
rP_vec =
        8000           0
vP1_vec =
         0    7.8542
rD_vec =
   1.0e+03 *
    0.0000    9.9048
vD1_vec =
   -6.3438    1.5104
tD =
   1.8731e+03
tC =
  542.7857
tPD =
   1.3304e+03
vP2_vec =
   -2.5294    9.5746
vD2_vec =
   -7.7333    4.3707
delvP =
   -2.5294    1.7204
delvD =
    1.3896   -2.8603
delV =
    6.2390
\end{matlab}
\end{prob}


\begin{prob}

\begin{matlab}
clear all;
close all;

RE=149.6e6;
RM=227.9e6;

muE=398600;
muM=42830;
muS=132.71e9;

mE=5.974e24;
mM=6.419e23;
mS=1.989e30;

% (a)
VA3=sqrt(2*muS/(RE+RM)*RE/RM)
VA2=sqrt(muS/RM)

% (b)
VD1=sqrt(muS/RE)
VD3=sqrt(2*muS/(RE+RM)*RM/RE)

% (c)
a3=1/2*(RE+RM);
T3=2*pi/sqrt(muS)*a3^(3/2);
tDA=T3/2

>> prob1
VA3 =
   21.4833
VA2 =
   24.1312
VD1 =
   29.7842
VD3 =
   32.7276
tDA =
   2.2363e+07
\end{matlab}
\end{prob}

\begin{prob}

\begin{matlab}
clear all;
close all;

RE=149.6e6;
RM=227.9e6;

muE=398600;
muM=42830;
muS=132.71e9;

mE=5.974e24;
mM=6.419e23;
mS=1.989e30;

thetaM=358.13*pi/180;
thetaE=230.81*pi/180;

TM=2*pi/sqrt(muS)*RM^(3/2);
TE=2*pi/sqrt(muS)*RE^(3/2);

nM=2*pi/TM
nE=2*pi/TE

% (a)
phi_t0=thetaM-thetaE
disp(nM-nE)

% (b)
a3=1/2*(RE+RM);
T3=2*pi/sqrt(muS)*a3^(3/2);
tDA=T3/2;
phi_0=pi-nM*tDA
disp(phi_0*180/pi);

% (c)
for k=1:6;
    td(k)=(phi_0-2*pi*k-phi_t0)/(nM-nE)/3600/24;
end
disp('td_k');
disp(td);

k=6
td=td(6)

% (d)
disp('Saturday, December 17, 2016');

>> prob2
nM =
   1.0589e-07
nE =
   1.9909e-07
phi_t0 =
    2.2222    
  -9.3207e-08
phi_0 =
    0.7737
   44.3292
td_k
   1.0e+03 *
    0.9601    1.7403    2.5205    3.3007    4.0810    4.8612
k =
     6
td =
   4.8612e+03
Saturday, December 17, 2016
\end{matlab}
\end{prob}

\begin{prob}
\begin{matlab}
clear all;
close all;

RE=149.6e6;
RM=227.9e6;

muE=398600;
muM=42830;
muS=132.71e9;

mE=5.974e24;
mM=6.419e23;
mS=1.989e30;

disp('(a)');
rSOI_E=(mE/mS)^(2/5)*RE

disp('(b)');
VD1=sqrt(muS/RE);
VD3=sqrt(2*muS/(RE+RM)*RM/RE);
v_inf=VD3-VD1

disp('(c)');
rp=9000;
h=rp*sqrt(v_inf^2+2*muE/rp)
e=1+rp*v_inf^2/muE

disp('(d)');
vp=h/rp;
vc=sqrt(muE/rp);
delv=vp-vc

disp('(e)');

beta=acos(1/e)
beta*180/pi
angle=beta+pi/2
angle*180/pi

clear all;
close all;

RE=149.6e6;
RM=227.9e6;

muE=398600;
muM=42830;
muS=132.71e9;

mE=5.974e24;
mM=6.419e23;
mS=1.989e30;

disp('(a)');
rSOI_E=(mE/mS)^(2/5)*RE

disp('(b)');
VD1=sqrt(muS/RE);
VD3=sqrt(2*muS/(RE+RM)*RM/RE);
v_inf=VD3-VD1

disp('(c)');
rp=9000;
h=rp*sqrt(v_inf^2+2*muE/rp)
e=1+rp*v_inf^2/muE

disp('(d)');
vp=h/rp;
vc=sqrt(muE/rp);
delv=vp-vc

disp('(e)');

beta=acos(1/e)
beta*180/pi
angle=beta+pi/2
angle*180/pi
\end{matlab}
\end{prob}

\begin{prob}
\begin{matlab}

clear all;
close all;

RE=149.6e6;
RM=227.9e6;

muE=398600;
muM=42830;
muS=132.71e9;

mE=5.974e24;
mM=6.419e23;
mS=1.989e30;

disp('(a)');
rSOI_M=(mM/mS)^(2/5)*RM
rSOI_M/RM

disp('(b)');
VA2=sqrt(muS/RM);
VA3=sqrt(2*muS/(RE+RM)*RE/RM);
v_inf=VA2-VA3

disp('(c)');
rp=3396;

h=rp*sqrt(v_inf^2+2*muM/rp)
e=1+rp*v_inf^2/muM

disp('(d)');
vp=h/rp

disp('(e)');
beta=acos(1/e)
beta*180/pi
angle=pi/2-beta
angle*180/pi

disp('(f)');
Delta=h^2/muM/sqrt(e^2-1)

>> prob4
(a)
rSOI_M =
   5.7713e+05
ans =
    0.0025
(b)
v_inf =
    2.6479
(c)
h =
   1.9281e+04
e =
    1.5559
(d)
vp =
    5.6776
(e)
beta =
    0.8728
ans =
   50.0064
angle =
    0.6980
ans =
   39.9936
(f)
Delta =
   7.2817e+03
\end{matlab}
\end{prob}

\end{document}

