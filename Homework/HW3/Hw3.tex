\documentclass[10pt]{article}
\usepackage[letterpaper,text={6.5in,8.7in},centering]{geometry}
\usepackage{amssymb,amsmath,times,url,subfigure,graphicx,theorem,alltt}
%\usepackage[pdftex,urlcolor=blue,pdfpagemode=none,pdfstartview=FitH]{hyperref}

%% url smaller font.
\makeatletter
\def\url@leostyle{%
  \@ifundefined{selectfont}{\def\UrlFont{\sf}}{\def\UrlFont{\small\ttfamily}}}
\makeatother
\urlstyle{leo}

%\usepackage[all,import]{xy}

\newcommand{\norm}[1]{\ensuremath{\left\| #1 \right\|}}
\newcommand{\abs}[1]{\ensuremath{\left| #1 \right|}}
\newcommand{\bracket}[1]{\ensuremath{\left[ #1 \right]}}
\newcommand{\braces}[1]{\ensuremath{\left\{ #1 \right\}}}
\newcommand{\parenth}[1]{\ensuremath{\left( #1 \right)}}
\newcommand{\ip}[1]{\ensuremath{\langle #1 \rangle}}
\newcommand{\refeqn}[1]{(\ref{eqn:#1})}
\newcommand{\reffig}[1]{Fig. \ref{fig:#1}}
\newcommand{\tr}[1]{\mbox{tr}\ensuremath{\negthickspace\bracket{#1}}}
\newcommand{\deriv}[2]{\ensuremath{\frac{\partial #1}{\partial #2}}}
\newcommand{\SO}{\ensuremath{\mathrm{SO(3)}}}
\newcommand{\T}{\ensuremath{\mathrm{T}}}
\newcommand{\so}{\ensuremath{\mathfrak{so}(3)}}
\newcommand{\SE}{\ensuremath{\mathrm{SE(3)}}}
\newcommand{\se}{\ensuremath{\mathfrak{se}(3)}}
\renewcommand{\Re}{\ensuremath{\mathbb{R}}}
\renewcommand{\S}{\ensuremath{\mathbb{S}}}
\newcommand{\aSE}[2]{\ensuremath{\begin{bmatrix}#1&#2\\0&1\end{bmatrix}}}
\newcommand{\ase}[2]{\ensuremath{\begin{bmatrix}#1&#2\\0&0\end{bmatrix}}}
\newcommand{\D}{\ensuremath{\mathbf{D}}}
\newcommand{\pair}[1]{\ensuremath{\left\langle #1 \right\rangle}}
\newcommand{\met}[1]{\ensuremath{\langle\!\langle #1 \rangle\!\rangle}}
\newcommand{\Ad}{\ensuremath{\mathrm{Ad}}}
\newcommand{\ad}{\ensuremath{\mathrm{ad}}}
\newcommand{\g}{\ensuremath{\mathfrak{g}}}

\renewcommand{\baselinestretch}{1.2}
\date{}

\renewcommand{\thesubsection}{\arabic{subsection}. }
\renewcommand{\thesubsubsection}{\arabic{subsection}.\arabic{subsubsection} }

\theoremstyle{plain}\theorembodyfont{\normalfont}
\newtheorem{prob}{Problem}[section]
%\renewcommand{\theprob}{\arabic{section}.\arabic{prob}}
\renewcommand{\theprob}{\arabic{prob}}

\newenvironment{subprob}%
{\renewcommand{\theenumi}{\alph{enumi}}\renewcommand{\labelenumi}{(\theenumi)}\begin{enumerate}}%
{\end{enumerate}}%

\newenvironment{matlab}
{\begin{alltt}\small\renewcommand{\baselinestretch}{1.2}\selectfont}%
{\end{alltt}}


\begin{document}

\pagestyle{empty}
\section*{MAE3145: Homework 3}
\vspace*{-0.4cm}
\noindent{Due date: October 19, 2016}%\\%\vspace*{0.5cm}


\begin{prob}
Consider Asteroid 5 discussed at Question 3 of HW\#2. Its specific energy and specific angular momentum are given by $\mathcal{E}=10\,\mathrm{km^2/s^3}$, and $h=8\times 10^4\,\mathrm{km^2/s}$. We want to determine the time after periapsis passage $t$ when the true anomaly is $\theta=100^\circ$.
\begin{subprob}
\item Compute the semi-major axis $a$, and the eccentricity $e$.
\item Compute the maximum true anomaly $\theta_\infty$. Is $\theta < \theta_\infty$?
\item Compute the hyperbolic eccentric anomaly $F$, and the hyperbolic mean anomaly $M_e$.
\item Show that the time after the periapsis passage is given by $t=0.6979\,\mathrm{hrs}$.
\end{subprob}
\end{prob}

\begin{prob}
An Earth-orbiting satellite has a period of $T=15.743$ hours and a periapsis radius $r_p=12756\,\mathrm{km}$. We want to determine the location of this satellite at time $t=1$ hour after periapsis passage.
\begin{subprob}
\item Compute the semi-major axis $a$, and the eccentricity $e$.
\item Compute the mean anomaly $M_e$.
\item Write a Matlab program to compute the eccentric anomaly $E$.
\item Show that the true anomaly is given by $\theta=84.2850^\circ$.
\end{subprob}
(Hint: if you want to verify your code, check that your code gives $\theta=\pi$ when $t=T/2$.)
\end{prob}

\begin{prob} We observed the position and the velocity of a spacecraft orbiting the Earth as follows:
\begin{align*}
\vec r_0 = [6000,6000,6000]\,\mathrm{km},\quad \vec v_0 = [-5,-5,0]\,\mathrm{km/s}.
\end{align*}
Assume that $\mu = 398600 \,\mathrm{km^3/s^2}$.

\begin{subprob}
\item Using the Matlab \texttt{rv2oe.m} posted to Blackboard, find the orbital elements $(h,e,\theta,\Omega,\omega,i)$.
\item Write a Matlab function \texttt{oe2rv.m} that computes the position and the velocity vector for  given orbital elements. The first few lines of this file are as follows:
\begin{matlab}
1: function [r_vec, v_vec]=oe2rv(h,e,theta,Omega,i,omega)
2: xhat=[1;0;0];
3: yhat=[0;1;0];
4: zhat=[0;0;1];
5: mu=398600;
6: N_hat=
\end{matlab}

(Hint: if you want to verify your code, check that your code returns $\vec r_0,\vec v_0$ when the orbital elements are equal to your answer to (a).)

\item Evaluate the function \texttt{oe2rv.m} for varying \texttt{theta=linspace(0,2*pi,200)}. The other five orbital elements $(h,e,\Omega,\omega,i)$ are fixed at your solution of (a). Plot the position and the velocity vector in a three-dimensional space. The structure of the Matlab file is as follows:
\begin{matlab}
 1: h=
 2: e=
 3: Omega=
 4: omega=
 5: i=
 6: theta=linspace(0,2*pi,200);
 7: for k=1:200
 8:      [r_vec_theta(:,k),v_vec_theta(:,k)]=oe2rv(h,e,theta(k),
 9: end
10: plot3(r_vec_theta(1,:),
11: figure;
12: plot3(v_vec_theta(1,:),
\end{matlab}
\item Check that $\vec r_0$ and $\vec v_0$ are on your curves at (c).
\end{subprob}
\end{prob}

%\clearpage\newpage
%\begin{prob}
%In a sun-synchronous orbit, the average precession rate of $\Omega$ is the same as the orbital angular rate of the Earth around the Sun:
%\begin{align}
%\dot{\overline \Omega} = -\frac{3}{2} \frac{\sqrt{\mu} J_2 R_E^2}{(1-e^2)^2 a^{7/2}}\cos i = 2\pi \,(\mathrm{rad/year}).
%\end{align}
%Assume that $\mu = 398600 \,\mathrm{km^3/s^2}$, $R_E=6378\,\mathrm{km}$, $J_2 = 0.00108$. In this problem, we consider circular sun-synchronous orbits, i.e. $e=0$.
%\begin{subprob}
%\item Find the maximum orbital radius $a_M$ for circular sun-synchronous orbits. (Note that the orbital radius $a$ is a function of the inclination $i$.)
%\item Show that the inclination $i$ and the orbital radius $a$ of circular sun-synchronous orbits has the following relation:
%\begin{align}
%\cos i = - \parenth{\frac{a}{a_M}}^{7/2}\label{eqn:i}
%\end{align}
%\item Using \refeqn{i}, plot a curve showing the inclination as a function of the orbital radius when $R_E\leq a \leq a_M$.
%\item Find the maximum and the minimum orbit period that a circular sun-synchronous orbit can have.
%\end{subprob}
%\end{prob}


\begin{prob} A satellite satisfies the following condition at the current time.
\begin{itemize}
\item $\vec r = [-6634.2, -1261.8, -5230.9]\,\mathrm{km}$
\item $\vec e = [-0.40907, -0.48751, -0.63640]$
\item It is flying toward its periapsis.
\end{itemize}
\begin{subprob}
\item What is the type of orbit.
\item Find the direction of the specific angular momentum $\hat h = \frac{\vec h}{h}$.
\item Find the inclination $i$. %See Exercise 4.5 at page 188. Is the given solution correct?
\item Find the direction of the node vector $\hat N = \frac{\vec N}{N}$.
\item Find the longitude of the ascending node $\Omega$.
\item Find the argument of periapsis $\omega$.
\item Find the true anomaly $\theta$.
\end{subprob}
\end{prob}

\end{document}

