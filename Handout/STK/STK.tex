\documentclass[10pt]{article}
\usepackage[letterpaper,text={6.5in,8.7in},centering]{geometry}
\usepackage{amssymb,amsmath,times,url,subfigure,graphicx,theorem,alltt,eepic,tikz}
%\usepackage[pdftex,urlcolor=blue,pdfpagemode=none,pdfstartview=FitH]{hyperref}

%% url smaller font.
\makeatletter
\def\url@leostyle{%
  \@ifundefined{selectfont}{\def\UrlFont{\sf}}{\def\UrlFont{\small\ttfamily}}}
\makeatother
\urlstyle{leo}

%\usepackage[all,import]{xy}

\newcommand{\norm}[1]{\ensuremath{\left\| #1 \right\|}}
\newcommand{\abs}[1]{\ensuremath{\left| #1 \right|}}
\newcommand{\bracket}[1]{\ensuremath{\left[ #1 \right]}}
\newcommand{\braces}[1]{\ensuremath{\left\{ #1 \right\}}}
\newcommand{\parenth}[1]{\ensuremath{\left( #1 \right)}}
\newcommand{\ip}[1]{\ensuremath{\langle #1 \rangle}}
\newcommand{\refeqn}[1]{(\ref{eqn:#1})}
\newcommand{\reffig}[1]{Fig. \ref{fig:#1}}
\newcommand{\tr}[1]{\mbox{tr}\ensuremath{\negthickspace\bracket{#1}}}
\newcommand{\deriv}[2]{\ensuremath{\frac{\partial #1}{\partial #2}}}
\newcommand{\SO}{\ensuremath{\mathrm{SO(3)}}}
\newcommand{\T}{\ensuremath{\mathrm{T}}}
\newcommand{\so}{\ensuremath{\mathfrak{so}(3)}}
\newcommand{\SE}{\ensuremath{\mathrm{SE(3)}}}
\newcommand{\se}{\ensuremath{\mathfrak{se}(3)}}
\renewcommand{\Re}{\ensuremath{\mathbb{R}}}
\renewcommand{\S}{\ensuremath{\mathbb{S}}}
\newcommand{\aSE}[2]{\ensuremath{\begin{bmatrix}#1&#2\\0&1\end{bmatrix}}}
\newcommand{\ase}[2]{\ensuremath{\begin{bmatrix}#1&#2\\0&0\end{bmatrix}}}
\newcommand{\D}{\ensuremath{\mathbf{D}}}
\newcommand{\pair}[1]{\ensuremath{\left\langle #1 \right\rangle}}
\newcommand{\met}[1]{\ensuremath{\langle\!\langle #1 \rangle\!\rangle}}
\newcommand{\Ad}{\ensuremath{\mathrm{Ad}}}
\newcommand{\ad}{\ensuremath{\mathrm{ad}}}
\newcommand{\g}{\ensuremath{\mathfrak{g}}}

\renewcommand{\baselinestretch}{1.2}
\date{}

\renewcommand{\thesubsection}{\arabic{subsection}. }
\renewcommand{\thesubsubsection}{\arabic{subsection}.\arabic{subsubsection} }

\theoremstyle{plain}\theorembodyfont{\normalfont}
\newtheorem{prob}{Problem}[section]
%\renewcommand{\theprob}{\arabic{section}.\arabic{prob}}
\renewcommand{\theprob}{\arabic{prob}}

\newenvironment{subprob}%
{\renewcommand{\theenumi}{\alph{enumi}}\renewcommand{\labelenumi}{(\theenumi)}\begin{enumerate}}%
{\end{enumerate}}%

\newenvironment{matlab}
{\begin{alltt}\small\renewcommand{\baselinestretch}{1.2}\selectfont}%
{\end{alltt}}

\newcommand*\circled[1]{%
  \tikz[baseline=(C.base)]\node[draw,circle,inner sep=0.5pt](C) {#1};\!
}

\renewcommand\emph[1]{\textsf{#1}}

\begin{document}

\pagestyle{empty}
\section*{MAE3145: Systems Took Kit (STK)}
\vspace*{-0.4cm}
\noindent{November 12, 2014}%\\%\vspace*{0.5cm}

\subsection{Spy on DC from a Molniya Satellite}

\paragraph{Goal} Generate ground track for a satellite on a Molniya orbit, and determine how long/often it can observe DC.


\paragraph{Create a New Scenario}
\begin{list}{$\bullet$}
{\setlength{\itemsep}{-3pt}\setlength{\leftmargin}{30pt}}
\item Run STK
\item Click  \emph{Create a New Scenario}
\item At the \emph{STK: New Scenario Wizard}, type \emph{WatchDC} for \emph{Name} 
\item[*] At the \emph{STK: New Scenario Wizard}, set \emph{Analysis Period} from \textcolor{red}{\emph{1 Jan 2014}} to \textcolor{red}{\emph{31 Jan 2014}}
%Students tend to make a mistake here.
\item Click \emph{Ok}, then the \emph{Insert STK Object} window appears
\end{list}



\paragraph{Model Spacecraft}

\begin{list}{$\bullet$}
{\setlength{\itemsep}{-3pt}\setlength{\leftmargin}{30pt}}
\item Choose \emph{Satellite} and \emph{Orbit Wizard}
\item Click \emph{Insert}, then the \emph{Orbit Wizard} window appears
\item Choose \emph{Orbit Designer} at the \emph{Type} menu, and type \emph{SpySat} at \emph{Satellite Name}
\item Choose \emph{Orbital Elements} as follows:
\textcolor{red}{
\begin{align*}
a=26610.21\,\mathrm{km},\quad e=0.7,\quad i=63.4^\circ,\quad 
\omega=270^\circ,\quad \Omega=230^\circ\,(\mbox{RAAN}),\quad \theta=0^\circ
\end{align*}}\vspace*{-0.5cm}
\item Click \emph{Okay} and close the \emph{Insert STK Object} window
\item Click \emph{Start} at the tool bars and watch \emph{2D Graphics Window} and \emph{3D Graphics Window}
\item At the \emph{3D Graphics Window,} drag to rotate the focal point, or drag while clicking the right button to zoom in/out.
\end{list}

\paragraph{Model Washington, DC}

\begin{list}{$\bullet$}
{\setlength{\itemsep}{-3pt}\setlength{\leftmargin}{30pt}}
\item In the menu, click \emph{Insert} and \emph{New}, then the \emph{Insert STK Objects} window appears
\item Choose \emph{Place} and \emph{Select From City Database}
\item Click \emph{Insert}, then the \emph{Insert From City Database} window appears
\item Type \emph{Washington} at City Name, and click \emph{Search}
\item Select Washington DC and click \emph{Insert}
\item Close the \emph{Insert STK Objects} window
\item Find Washington from the \emph{2D Graphics Window} and the \emph{3D Graphics Window}
\end{list}

\paragraph{Analyze Observation Time}

\begin{list}{$\bullet$}
{\setlength{\itemsep}{-3pt}\setlength{\leftmargin}{30pt}}
\item In the menu, click \emph{Analysis} and \emph{Access}, then the \emph{Access} window appears
\item Check that the item for \emph{Access For} is your \emph{SpySat}
\item Choose \emph{Washington} from \emph{Associated Objects} and click \emph{Compute}
\item Click \emph{Close}
\item Watch the \emph{2D Graphics Window} and the \emph{3D Graphics Window}
\end{list}

\paragraph{Create an Access Report}

\begin{list}{$\bullet$}
{\setlength{\itemsep}{-3pt}\setlength{\leftmargin}{30pt}}
\item In the menu, click \emph{Analysis} and \emph{Access}, then the \emph{Access} window appears
\item Click \emph{Access..} at the \emph{Reports} group, then the report window appears
\item Click the icon for \emph{Global Access Units}, then the \emph{Units: Access} window appears
\item Select \emph{Hours} and click \emph{Ok}
\item Click the icon for \emph{Save As}, and type the file name and save it
\item \textcolor{red}{Upload your txt report to Black Board }

\end{list}

\subsection{Hohmann Transfer}

\paragraph{Goal} A satellite is on a circular orbit with the orbital radius of $r_P=7000\,\mathrm{km}$. We wish to design an impulsive maneuver such that the orbital radius is increased to $r_A=9000\,\mathrm{km}$. The required velocity changes are computed as follows:
\begin{align*}
\Delta v_P &= \sqrt{\frac{2\mu}{r_A+r_P}\frac{r_A}{r_P}} - \sqrt{\frac{\mu}{r_P}} = \text{\textcolor{red}{$0.4577\,\mathrm{km/s}$}},\\
\Delta v_A &= \sqrt{\frac{\mu}{r_A}}- \sqrt{\frac{2\mu}{r_A+r_P}\frac{r_P}{r_A}}  = \text{\textcolor{red}{$0.4298\,\mathrm{km/s}$}}.
\end{align*}

The window for the previous example should be closed. Otherwise, there is an error for license.

\paragraph{First Impulse}

\begin{list}{$\bullet$}
{\setlength{\itemsep}{-3pt}\setlength{\leftmargin}{30pt}}
\item Download \emph{Hohmann.zip} from Black Board, and extract it to a folder.

(\textcolor{red}{Note: You SHOULD EXTRACT all of the files to your folder!} Double-clicking the scenario file from the zip folder prevents loading predefined satellite.)

\item Open \emph{Hohmann.sc}
\item Double-click \emph{GWU\_Sat}, then the \emph{Basic Orbit} window appears
\item Click the icon for \emph{Insert Segment}, then the \emph{Segment Section} window appears
\item Select \emph{Maneuver} and click \emph{Ok}
\item Type \textcolor{red}{\emph{0.4577}} at \emph{Delta V Magnitude} and click \emph{Apply}
\item Click the icon for \emph{Insert Segment}, then the \emph{Segment Section} window appears
\item Select \emph{Propagate} and click \emph{Ok}
\item Uncheck \emph{Duration}
\item Click \emph{Insert} and choose \emph{Apoapsis}
\item Click the icon for \emph{Run Entire Mission Control Sequence}
\item Watch the \emph{2D Graphics Window} and the \emph{3D Graphics Window}
\end{list}

\paragraph{Second Impulse}

\begin{list}{$\bullet$}
{\setlength{\itemsep}{-3pt}\setlength{\leftmargin}{30pt}}
\item Double-click \emph{GWU\_Sat}, then the \emph{Basic Orbit} window appears
\item Click the icon for \emph{Insert Segment}, then the \emph{Segment Section} window appears
\item Select \emph{Maneuver} and click \emph{Ok}
\item Type \textcolor{red}{\emph{0.4298}} at \emph{Delta V Magnitude} and click \emph{Apply}
\item Click the icon for \emph{Insert Segment}, then the \emph{Segment Section} window appears
\item Select \emph{Propagate} and click \emph{Ok}
\item Click the icon for \emph{Run Entire Mission Control Sequence}
\item Watch the \emph{2D Graphics Window} and the \emph{3D Graphics Window}
\item Click the icon for \emph{Snap Frame} at the \emph{3D Graphics Window}, and type the file name and save it as jpg
\item \textcolor{red}{Upload your jpg file to Black Board }
\end{list}

\end{document}

