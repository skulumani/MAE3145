\documentclass[10pt]{article}
\usepackage[letterpaper,text={6.5in,8.7in},centering]{geometry}
\usepackage{amssymb,amsmath,times,url,subfigure,graphicx,theorem,alltt}
%\usepackage[pdftex,urlcolor=blue,pdfpagemode=none,pdfstartview=FitH]{hyperref}

%% url smaller font.
\makeatletter
\def\url@leostyle{%
  \@ifundefined{selectfont}{\def\UrlFont{\sf}}{\def\UrlFont{\small\ttfamily}}}
\makeatother
\urlstyle{leo}

%\usepackage[all,import]{xy}

\newcommand{\norm}[1]{\ensuremath{\left\| #1 \right\|}}
\newcommand{\abs}[1]{\ensuremath{\left| #1 \right|}}
\newcommand{\bracket}[1]{\ensuremath{\left[ #1 \right]}}
\newcommand{\braces}[1]{\ensuremath{\left\{ #1 \right\}}}
\newcommand{\parenth}[1]{\ensuremath{\left( #1 \right)}}
\newcommand{\ip}[1]{\ensuremath{\langle #1 \rangle}}
\newcommand{\refeqn}[1]{(\ref{eqn:#1})}
\newcommand{\reffig}[1]{Fig. \ref{fig:#1}}
\newcommand{\tr}[1]{\mbox{tr}\ensuremath{\negthickspace\bracket{#1}}}
\newcommand{\deriv}[2]{\ensuremath{\frac{\partial #1}{\partial #2}}}
\newcommand{\SO}{\ensuremath{\mathrm{SO(3)}}}
\newcommand{\T}{\ensuremath{\mathrm{T}}}
\newcommand{\so}{\ensuremath{\mathfrak{so}(3)}}
\newcommand{\SE}{\ensuremath{\mathrm{SE(3)}}}
\newcommand{\se}{\ensuremath{\mathfrak{se}(3)}}
\renewcommand{\Re}{\ensuremath{\mathbb{R}}}
\renewcommand{\S}{\ensuremath{\mathbb{S}}}
\newcommand{\aSE}[2]{\ensuremath{\begin{bmatrix}#1&#2\\0&1\end{bmatrix}}}
\newcommand{\ase}[2]{\ensuremath{\begin{bmatrix}#1&#2\\0&0\end{bmatrix}}}
\newcommand{\D}{\ensuremath{\mathbf{D}}}
\newcommand{\pair}[1]{\ensuremath{\left\langle #1 \right\rangle}}
\newcommand{\met}[1]{\ensuremath{\langle\!\langle #1 \rangle\!\rangle}}
\newcommand{\Ad}{\ensuremath{\mathrm{Ad}}}
\newcommand{\ad}{\ensuremath{\mathrm{ad}}}
\newcommand{\g}{\ensuremath{\mathfrak{g}}}

\renewcommand{\baselinestretch}{1.2}
\date{}

\renewcommand{\thesubsection}{\arabic{subsection}. }
\renewcommand{\thesubsubsection}{\arabic{subsection}.\arabic{subsubsection} }

\theoremstyle{plain}\theorembodyfont{\normalfont}
\newtheorem{prob}{Problem}[section]
%\renewcommand{\theprob}{\arabic{section}.\arabic{prob}}
\renewcommand{\theprob}{\arabic{prob}}

\newenvironment{subprob}%
{\renewcommand{\theenumi}{\alph{enumi}}\renewcommand{\labelenumi}{(\theenumi)}\begin{enumerate}}%
{\end{enumerate}}%


\begin{document}

\pagestyle{empty}
\section*{Transformation between Orbital Elements and $(\vec r, \vec v)$}

\subsubsection*{Given $(\vec r, \vec v)$, find the orbital elements $(h,e,\theta,\Omega,i,\omega)$}

\begin{align*}
r & = |\vec r|,\\ 
\vec h & = \vec r \times \vec v,\qquad h = |\vec h|,\\
\vec e & = \frac{1}{\mu} \vec v \times \vec h - \frac{\vec r}{r},\qquad e = | \vec e|,\\
\vec N & = \hat z \times \vec h,\\
i & = \cos^{-1} \parenth{\frac{\vec h\cdot \hat z}{h}}\quad (0\leq i \leq \pi),\\
\Omega & = \tan^{-1} \parenth{\frac{\hat y \cdot \vec N}{\hat x \cdot \vec N}}
=\texttt{numpy.arctan2(}{{\hat y \cdot \vec N}\texttt{,}{\hat x \cdot \vec N}}\texttt{)},\\
\omega & = \tan^{-1} \parenth{\frac{\vec h \cdot (\vec N \times \vec e)}{h (\vec N \cdot \vec e)}}
=\texttt{numpy.arctan2(}{{\vec h \cdot (\vec N \times \vec e)}\texttt{,}{h (\vec N \cdot \vec e)}}\texttt{)},\\
\theta & = \tan^{-1} \parenth{\frac{\vec h \cdot (\vec e \times \vec r)}{h (\vec e \cdot \vec r)}}=\texttt{numpy.arctan2(}{{\vec h \cdot (\vec e \times \vec r)}\texttt{,}{h (\vec e \cdot \vec r)}}\texttt{)}.
\end{align*}

\noindent (Use the Numpy \texttt{numpy.arctan2} function to compute $\tan^{-1}$, i.e. $\tan^{-1}(y/x) =\texttt{numpy.arctan2(y,x)}$).\\


\subsubsection*{Given the orbital elements $(h,e,\theta,\Omega,i,\omega)$, find $(\vec r, \vec v)$}

\begin{align*}
\hat N & = \cos\Omega\, \hat x + \sin\Omega\, \hat y,\\
\hat h & = \sin i \sin\Omega \,\hat x - \sin i \cos\Omega \,\hat y + \cos i \,\hat z,\\
\hat N_t & = -\sin\Omega\cos i \,\hat x +\cos\Omega \cos i \,\hat y + \sin i \,\hat z,\\
\hat u_r & = \cos(\theta+\omega) \,\hat N + \sin(\theta+\omega) \,\hat N_t,\\
\hat u_\theta & = -\sin(\theta+\omega) \,\hat N + \cos(\theta+\omega)\, \hat N_t,\\
r & = \frac{h^2}{\mu} \frac{1}{1+e\cos\theta},\\
\mathcal{E} & = -\frac{1}{2}\frac{\mu^2}{h^2} (1-e^2),\\
v & = \sqrt{2\parenth{\mathcal{E} + \frac{\mu}{r}}},\\
\gamma & = \tan^{-1} \parenth{\frac{e\sin\theta}{1+e\cos\theta}},\\
\vec r & = r \,\hat u_r,\\
\vec v & = v \cos\gamma \,\hat u_\theta + v\sin\gamma \,\hat u_r.
\end{align*}

\end{document}

