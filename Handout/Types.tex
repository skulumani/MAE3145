\documentclass[10pt]{article}
\usepackage[letterpaper,text={6.5in,8.7in},centering]{geometry}
\usepackage{amssymb,amsmath,times,url,subfigure,graphicx,theorem,alltt}
%\usepackage[pdftex,urlcolor=blue,pdfpagemode=none,pdfstartview=FitH]{hyperref}

%% url smaller font.
\makeatletter
\def\url@leostyle{%
  \@ifundefined{selectfont}{\def\UrlFont{\sf}}{\def\UrlFont{\small\ttfamily}}}
\makeatother
\urlstyle{leo}

%\usepackage[all,import]{xy}

\newcommand{\norm}[1]{\ensuremath{\left\| #1 \right\|}}
\newcommand{\abs}[1]{\ensuremath{\left| #1 \right|}}
\newcommand{\bracket}[1]{\ensuremath{\left[ #1 \right]}}
\newcommand{\braces}[1]{\ensuremath{\left\{ #1 \right\}}}
\newcommand{\parenth}[1]{\ensuremath{\left( #1 \right)}}
\newcommand{\ip}[1]{\ensuremath{\langle #1 \rangle}}
\newcommand{\refeqn}[1]{(\ref{eqn:#1})}
\newcommand{\reffig}[1]{Fig. \ref{fig:#1}}
\newcommand{\tr}[1]{\mbox{tr}\ensuremath{\negthickspace\bracket{#1}}}
\newcommand{\deriv}[2]{\ensuremath{\frac{\partial #1}{\partial #2}}}
\newcommand{\SO}{\ensuremath{\mathrm{SO(3)}}}
\newcommand{\T}{\ensuremath{\mathrm{T}}}
\newcommand{\so}{\ensuremath{\mathfrak{so}(3)}}
\newcommand{\SE}{\ensuremath{\mathrm{SE(3)}}}
\newcommand{\se}{\ensuremath{\mathfrak{se}(3)}}
\renewcommand{\Re}{\ensuremath{\mathbb{R}}}
\renewcommand{\S}{\ensuremath{\mathbb{S}}}
\newcommand{\aSE}[2]{\ensuremath{\begin{bmatrix}#1&#2\\0&1\end{bmatrix}}}
\newcommand{\ase}[2]{\ensuremath{\begin{bmatrix}#1&#2\\0&0\end{bmatrix}}}
\newcommand{\D}{\ensuremath{\mathbf{D}}}
\newcommand{\pair}[1]{\ensuremath{\left\langle #1 \right\rangle}}
\newcommand{\met}[1]{\ensuremath{\langle\!\langle #1 \rangle\!\rangle}}
\newcommand{\Ad}{\ensuremath{\mathrm{Ad}}}
\newcommand{\ad}{\ensuremath{\mathrm{ad}}}
\newcommand{\g}{\ensuremath{\mathfrak{g}}}

\renewcommand{\baselinestretch}{1.2}
\date{}

\renewcommand{\thesubsection}{\arabic{subsection}. }
\renewcommand{\thesubsubsection}{\arabic{subsection}.\arabic{subsubsection} }

\theoremstyle{plain}\theorembodyfont{\normalfont}
\newtheorem{prob}{Problem}[section]
%\renewcommand{\theprob}{\arabic{section}.\arabic{prob}}
\renewcommand{\theprob}{\arabic{prob}}

\newenvironment{subprob}%
{\renewcommand{\theenumi}{\alph{enumi}}\renewcommand{\labelenumi}{(\theenumi)}\begin{enumerate}}%
{\end{enumerate}}%


\begin{document}


\section*{Summary of Orbital Properties}
\pagestyle{empty}
\subsection*{Any Type of Orbit}

\begin{alignat*}{3}
\vec h & = \vec r \times \vec v,&\qquad 
\vec r & = r \hat u_r&\qquad
\vec v & = v_r \hat u_r + v_\theta \hat u_\theta,\\
h & = rv\cos\gamma = rv_\theta = r^2\dot\theta, &
r & = \frac{h^2/\mu}{1+e \cos\theta}, &
v_r & = \frac{\mu}{h}e\sin\theta = \dot r,\\
\vec e & = \frac{\vec v\times\vec h}{\mu}- \frac{\vec r}{r}, &
r_p & = \frac{h^2/\mu}{1+e}, &
v_\theta & = \frac{\mu}{h}(1+e\cos\theta) = r\dot\theta,\\
&&
\mathcal{E} & = \frac{1}{2} v^2 - \frac{\mu}{r} = -\frac{1}{2}\frac{\mu^2}{h^2}(1-e^2),&
\tan \gamma & = \frac{v_r}{v_\theta} = \frac{e\sin\theta}{1+e\cos\theta}.
\end{alignat*}

\subsection*{Circular Orbits: ($e=0$)}

\begin{alignat*}{3}
v & = \sqrt{\frac{\mu}{r}},&\qquad
\mathcal{E} & = -\frac{\mu}{2r},&\qquad
T & = \frac{2\pi}{\sqrt{\mu}} r^{3/2}.
\end{alignat*}

\subsection*{Elliptic Orbits: ($0<e<1$)}

\begin{alignat*}{3}
r & = \frac{a(1-e^2)}{1+e\cos\theta},&\qquad
a & = \frac{h^2/\mu}{1-e^2} =\frac{1}{2}(r_a+r_p),&\qquad
T & = \frac{2\pi}{\sqrt{\mu}} a^{3/2},\\
r_p & = \frac{h^2/\mu}{1+e} = a(1-e),&
b & = a \sqrt{1-e^2},&
e & = \frac{r_a-r_p}{r_a+r_p},\\
r_a & = \frac{h^2/\mu}{1-e} = a (1+e),&
\mathcal{E} & = -\frac{\mu}{2a},&
h & = \sqrt{\mu a(1-e^2)}.
\end{alignat*}

\subsection*{Parabolic Orbits: ($e=1$)}
\begin{alignat*}{2}
v & = \sqrt{\frac{2\mu}{r}},&\qquad
\mathcal{E} & = 0.
\end{alignat*}

\subsection*{Hyperbolic Orbits: ($e>1$)}

\begin{align*}
r & = \frac{a(e^2-1)}{1+e\cos\theta},&\qquad
a & = \frac{h^2/\mu}{e^2-1},&\qquad
\theta_\infty & = \cos^{-1} (-1/e),\\
r_p & = a(e-1),&
b & = a \sqrt{e^2-1},&
\beta & = \cos^{-1} (1/e),\\
&&
\mathcal{E} & = \frac{\mu}{2a},&
h & = \sqrt{\mu a(e^2-1)}.
\end{align*}





\end{document}

