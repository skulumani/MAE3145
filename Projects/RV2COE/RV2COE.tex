\documentclass[11pt, reqno]{article}    % use "amsart" instead of "article" for AMSLaTeX format
\usepackage{my_packages}
\usepackage{tikz_packages}
\usepackage[american,siunitx]{circuitikz}
\usepackage{pgfplots}
\pgfplotsset{compat=1.14}
\usepackage[explicit]{titlesec}


\tikzset{
    state/.style={
        rectangle,
        rounded corners,
        draw=black, very thick,
        minimum height=2em,
        inner sep=2pt,
        text centered,
        text width=3cm,
    },
}

\title{MAE 3145: COMFIX}
\author{Shankar Kulumani}
\date{Fall 2017}                          % Activate to display a given date or no date

\begin{document}
\begin{center}
{\Large \textbf{MAE3145: RV2COE}}
\end{center}
\subsection*{Description}

Your objective is to write a program which determines the classical orbital elements of an Earth-orbiting body given its position and velocity vectors in the Earth-Centered Inertial frame.
It is also required to compute other orbital parameters including, radius of perigee, radius of apogee, energy and period.

\subsection*{Program Specifications}
The following is a description of the inputs and outputs you your program:

\noindent \textbf{INPUTS:}
\begin{itemize}
    \item \( R \in \R^{3} \) - the components of the position vector in the Earth Centered Inertial frame given in kilometers (\si{\kilo\meter})
    \item \( V \in \R^{3} \) - the components of the velocity vector in the Earth Centered Inertial frame given in kilometers per second ( \si{\kilo\meter\per\second})
\end{itemize}

\noindent \textbf{OUTPUTS:}
\begin{itemize}
    \item \( a \)  - semimajor axis in kilometers (\si{\kilo\meter})
    \item \( e \) - eccentricity (unitless)
    \item \( i \) - inclination in radians (\si{\radian})
    \item \( \Omega \) - right ascension of the ascending node in radians (\si{\radian})
    \item \( \omega \) - arguement of perigee in radians (\si{\radian})
    \item \( \nu \) - true anomaly in radians (\si{\radian})
\end{itemize}
Plus other outputs which are listed later in the description.

\subsection*{Code Organization} 
Your program should be modular in constructions. 
This means you have seperate functions which perform \textbf{specific} tasks.
You \textbf{DO NOT} use global variables but rather pass data between various functions.
One suggested organization is given:

\begin{tikzpicture}[->, >=stealth']
    \tikzstyle{arrow}=[draw, -Latex];
    \node[state] (Driver) {\textbf{Driver} \\ Script};
    \node[state, below left = 2cm and 2cm of Driver, anchor=center] (Constants) {\textbf{Constants} \\ (R) Function};
    \node[state, below = 2cm of Driver,  anchor=center] (Input) {\textbf{Input} \\  (O) Function};
    \node[state, below right = 2cm and 2cm of Driver, anchor=center] (RV2COE) {\textbf{RV2COE} \\ (R) Function};
    \node[state, below = 2cm of RV2COE, anchor=center] (OrbParams) {\textbf{OrbParams}\\(R) Function};
    \node[state, right = 2cm of RV2COE, anchor = center] (Output) {\textbf{Output} \\ (O) Function};

    \path [arrow] (Driver.south) -- (Constants.north);
    \path [arrow] (Driver.south) -- (Input.north);
    \path [arrow] (Driver.south) -- (RV2COE.north);
    \path [arrow] (RV2COE.south) -- (OrbParams.north);
    \path [arrow] (Driver.south) -- (Output.north);
\end{tikzpicture}

\subsection*{Required Function Description}

The following functions must be written and accept the given inputs and provide the required outputs. 
Each function should be well tested, with unit tests, and well-documented following the class example.

\subsubsection*{\texttt{RV2COE}}
This function determines an objects six classical orbital elements from its position and velocity vectors.

\noindent{\textbf{Input:}}


\end{document}  
