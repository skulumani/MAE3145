\documentclass[11pt, reqno]{article}    % use "amsart" instead of "article" for AMSLaTeX format
\usepackage{my_packages}
\usepackage{tikz_packages}
\usepackage[american,siunitx]{circuitikz}
\usepackage{pgfplots}
\pgfplotsset{compat=1.14}
\usepackage[explicit]{titlesec}


\tikzset{
    state/.style={
        rectangle,
        rounded corners,
        draw=black, very thick,
        minimum height=2em,
        inner sep=2pt,
        text centered,
        text width=3cm,
    },
}

\title{MAE 3145: COMFIX}
\author{Shankar Kulumani}
\date{Fall 2017}                          % Activate to display a given date or no date

\begin{document}
\begin{center}
{\Large \textbf{MAE3145: RV2COE}}
\end{center}
\subsection*{Description}

Your objective is to write a program which determines the classical orbital elements of an Earth-orbiting body given its position and velocity vectors in the Earth-Centered Inertial frame.
It is also required to compute other orbital parameters including, radius of perigee, radius of apogee, energy and period.

\subsection*{Project Requirements}
After completing the project you must submit the hard copies of your work:
\begin{itemize}
    \item Complete algorithms for the main driver script as well as a seperate algorithm for each sub-function that you develop.
        Someone totally unfamiliar with astrodynamics should be able to duplicate your program in any computer language.
    \item Clear, concise and properly documented and tested code
    \item Correct outputs from your program which matches the test cases
    \item Any additional test cases you may have used. 
        Explain why you did or did not use any additional test cases.
\end{itemize}

\subsection*{Authorized Resources}
You may consult with your instructor, the course notes or other reference material, and other students. 
However, you \textbf{MAY NOT} copy another student or any other individuals code. 
The program you develop must be your own work.

\subsection*{Program Specifications}
The following is a description of the inputs and outputs you your program:

\noindent \textbf{INPUTS:}
\begin{itemize}
    \item \( R \in \R^{3} \) - the components of the position vector in the Earth Centered Inertial frame given in kilometers (\si{\kilo\meter})
    \item \( V \in \R^{3} \) - the components of the velocity vector in the Earth Centered Inertial frame given in kilometers per second ( \si{\kilo\meter\per\second})
\end{itemize}

\noindent \textbf{OUTPUTS:}
\begin{itemize}
    \item \( a \)  - semimajor axis in kilometers (\si{\kilo\meter})
    \item \( e \) - eccentricity (unitless)
    \item \( i \) - inclination in radians (\si{\radian})
    \item \( \Omega \) - right ascension of the ascending node in radians (\si{\radian})
    \item \( \omega \) - arguement of perigee in radians (\si{\radian})
    \item \( \nu \) - true anomaly in radians (\si{\radian})
\end{itemize}
Plus other outputs which are listed later in the description.

\subsection*{Code Organization} 
Your program should be modular in constructions. 
This means you have seperate functions which perform \textbf{specific} tasks.
Furthermore, each function should be tested and properly documented following the example shown in class.
You \textbf{MAY NOT}  use global variables but rather must pass data between various functions.
One suggested organization is given:

\begin{tikzpicture}[->, >=stealth']
    \tikzstyle{arrow}=[draw, -Latex];
    \node[state] (Driver) {\textbf{Driver} \\ Script};
    \node[state, below left = 2cm and 2cm of Driver, anchor=center] (Constants) {\textbf{Constants} \\ (R) Function};
    \node[state, below = 2cm of Driver,  anchor=center] (Input) {\textbf{Input} \\  (O) Function};
    \node[state, below right = 2cm and 2cm of Driver, anchor=center] (RV2COE) {\textbf{RV2COE} \\ (R) Function};
    \node[state, below = 2cm of RV2COE, anchor=center] (OrbParams) {\textbf{OrbParams}\\(R) Function};
    \node[state, right = 2cm of RV2COE, anchor = center] (Output) {\textbf{Output} \\ (O) Function};

    \path [arrow] (Driver.south) -- (Constants.north);
    \path [arrow] (Driver.south) -- (Input.north);
    \path [arrow] (Driver.south) -- (RV2COE.north);
    \path [arrow] (RV2COE.south) -- (OrbParams.north);
    \path [arrow] (Driver.south) -- (Output.north);
\end{tikzpicture}

\subsection*{Required Function Description}

The following functions must be written and accept the given inputs and provide the required outputs. 
Each function should be well tested, with unit tests, and well-documented following the class example.

\subsubsection*{\texttt{RV2COE}}
This function determines an objects six classical orbital elements from its position and velocity vectors.

\noindent{\textbf{Input:}}
\begin{itemize}
    \item \( R \in \R^{3} \) - the components of the position vector in the Earth Centered Inertial frame given in kilometers (\si{\kilo\meter})
    \item \( V \in \R^{3} \) - the components of the velocity vector in the Earth Centered Inertial frame given in kilometers per second ( \si{\kilo\meter\per\second})
\end{itemize}

\noindent \textbf{OUTPUTS:}
\begin{itemize}
    \item \( a \)  - semimajor axis in kilometers (\si{\kilo\meter})
    \item \( e \) - eccentricity (unitless)
    \item \( i \) - inclination in radians (\si{\radian})
    \item \( \Omega \) - right ascension of the ascending node in radians (\si{\radian})
    \item \( \omega \) - arguement of perigee in radians (\si{\radian})
    \item \( \nu \) - true anomaly in radians (\si{\radian})
\end{itemize}

\subsection*{Optional Functions}
The following functions may be useful in the development of your program. 
It can allow you to further organize the code, but it is not required.

\subsubsection*{\texttt{hnevec}}
This function determines the angular momentum, ascending node, and eccentricity vectors given the position and velocity vectors.

\noindent{\textbf{Input:}}
\begin{itemize}
    \item \( R \in \R^{3} \) - the components of the position vector in the Earth Centered Inertial frame given in kilometers (\si{\kilo\meter})
    \item \( V \in \R^{3} \) - the components of the velocity vector in the Earth Centered Inertial frame given in kilometers per second ( \si{\kilo\meter\per\second})
\end{itemize}

\noindent \textbf{OUTPUTS:}
\begin{itemize}
    \item \( H \in \R^{3} \) - the components of specific angular momentum vector in the Earth Centered Inertial frame given in kilometers per second squared (\si{\kilo\meter\per\second\squared})
    \item \( N \in \R^{3} \) - the components of the ascending node vector in the Earth Centered Inertial frame given in kilometers per second squared (\si{\kilo\meter\per\second\squared})
    \item \( E \in \R^{3} \) - the components of the eccentricity vector in the Earth Centered Inertial frame (unitless)
\end{itemize}

\subsubsection*{\texttt{angles}}

This function determines all of the angular orbital elements. 
Ensure that you perform quadrant checks on all of the angles.

\noindent{\textbf{Input:}}
\begin{itemize}
    \item \( R \in \R^{3} \) - the components of the position vector in the Earth Centered Inertial frame given in kilometers (\si{\kilo\meter})
    \item \( V \in \R^{3} \) - the components of the velocity vector in the Earth Centered Inertial frame given in kilometers per second ( \si{\kilo\meter\per\second})
    \item \( H \in \R^{3} \) - the components of specific angular momentum vector in the Earth Centered Inertial frame given in kilometers per second squared (\si{\kilo\meter\per\second\squared})
    \item \( N \in \R^{3} \) - the components of the ascending node vector in the Earth Centered Inertial frame given in kilometers per second squared (\si{\kilo\meter\per\second\squared})
    \item \( E \in \R^{3} \) - the components of the eccentricity vector in the Earth Centered Inertial frame (unitless)
\end{itemize}

\noindent \textbf{OUTPUTS:}
\begin{itemize}
    \item \( i \) - inclination in radians (\si{\radian})
    \item \( \Omega \) - right ascension of the ascending node in radians (\si{\radian})
    \item \( \omega \) - arguement of perigee in radians (\si{\radian})
    \item \( \nu \) - true anomaly in radians (\si{\radian})
\end{itemize}

\subsubsection*{\texttt{orbitalparameters}}
This function determines other useful orbital parameters associated with the orbit.

\noindent \textbf{Input:}
\begin{itemize}
    \item \( a \)  - semimajor axis in kilometers (\si{\kilo\meter})
    \item \( e \) - eccentricity (unitless)
\end{itemize}

\noindent \textbf{Outputs:}
\begin{itemize}
    \item \( r_p \) - radius of perigee in kilometer (\si{\kilo\meter})
    \item \( r_a \) - radius of apogee in kilometer (\si{\kilo\meter})
    \item \( sme \) - specific mechanical energy in kilometer (\si{\kilo\meter})
    \item \( P \) - period of the orbit in seconds (\si{\second})
\end{itemize}

\subsubsection*{Input and Output}

You may consider creating seperate functions which can read and write to various text files. 
These are encouraged but are optional.

\subsection*{Test Data}
Two test cases are provided with this project. 
There is a text file named \texttt{RV1.txt} which contains the position and velocity vectors of two spacecraft. 
The data is copied below.
\begin{verbatim}
8840.0  646.0 5455.0  -0.695  5.25  -1.65
-3084.7  30.0  6911.0   5.66  -4.07   3.84 
\end{verbatim}
The numbers on each line corresponds to the three elements of the position vector and the three elements of the velocity vector, i.e. \(\begin{bmatrix} x & y & z & \dot{x} & \dot{y} & \dot{z} \end{bmatrix}\).
The solution for Case 1, the first line/spacecraft, is given below.
\begin{verbatim}
R (km)    =  8840.0000 I    646.0000 J   5455.0000 K    Mag= 10407.6866
V (km/s)  =    -0.6950 I      5.2500 J     -1.6500 K    Mag=     5.5469

Radius of Perigee           (km) =  6260.5311
Radius of Apogee            (km) = 11134.4744
Energy              (km^2/sec^2) =   -22.9147
Period                   (hours) =     2.2423
Semimajor Axis              (km) =  8697.5027
Eccentricity                     =     0.2802

Inclination                           (deg) =    33.9987
Right Ascension of the Ascending Node (deg) =   250.0287
Argument of Perigee                   (deg) =   255.5372
True Anomaly                          (deg) =   214.8548
\end{verbatim}

Using your computer program, you should read the vectors, compute the appropriate values and write them to a seperate text file.
Do not hardcode the data into your program, but rather read them from the file. 
While only two cases are given here, your program should be able to convert an arbitrary amount of position and velocity vectors.


\end{document}  
