\documentclass[11pt, reqno]{article}    % use "amsart" instead of "article" for AMSLaTeX format
\usepackage{my_packages}
\usepackage{tikz_packages}
\usepackage[american,siunitx]{circuitikz}
\usepackage{pgfplots}
\pgfplotsset{compat=1.14}
\usepackage[explicit]{titlesec}


\title{MAE 3145: COMFIX}
\author{Shankar Kulumani}
\date{Fall 2017}                          % Activate to display a given date or no date

\begin{document}
\begin{center}
{\Large \textbf{MAE3145: COMFIX}}
\end{center}
\subsection*{Description}
Your first mission as an orbital analyst has reached your capable hands.
Your task is to develop a program to convert radar observation data into Classical Orbital Elements (COEs).

Each radar observation includes the tracking site geodetic latitude, longitude, and its altitude above sea level.
The sites are equipped to measure vehicle range, pointing angles, Doppler shift, angular rates and record the observations in Universal Time (UTC).
The tracking data is reported as:
\begin{itemize}
    \item \textbf{LINE 1}
        \begin{itemize}
            \item Site latitude ( North is positive and South is Negative) in degrees
            \item Site longitude (East is positive and West is Negative) in degrees
            \item Site altitude in meters
            \item Observation time in UTC - \href{https://en.wikipedia.org/wiki/ISO_8601}{ISO 8601} format
        \end{itemize}
    \item \textbf{LINE 1}
        \begin{itemize}
            \item Satellite ID number - 4 digit integer
            \item Vehicle range in kilometers
            \item Vehicle azimuth in degrees
            \item Vehicle elevation in degrees
            \item Range rate in kilometers per second
            \item Azimuth rate in degrees per second
            \item Elevation rate in degrees per second \si{\degree\per\second}
        \end{itemize}
\end{itemize}

\subsection*{Project Requirements}
After completing the project you must submit the hard copies of your work:
\begin{itemize}
    \item Complete algorithms for the main driver script as well as a seperate algorithm for each sub-function that you develop.
        Someone totally unfamiliar with astrodynamics should be able to duplicate your program in any computer language.
    \item Clear, concise and properly documented and tested code
    \item Correct outputs from your program which matches the test cases
    \item Any additional test cases you may have used. 
        Explain why you did or did not use any additional test cases.
\end{itemize}

\subsection*{Authorized Resources}
You may consult with your instructor, the course notes or other reference material, and other students. 
However, you \textbf{MAY NOT} copy another student or any other individuals code. 
The program you develop must be your own work.

\subsection*{Algorithm}

Write a structured algorithm that shows your approach to writing a computer program to perform all of the tasks described above.
This should be a complete \textbf{sequential} list of the \textbf{equations and logic (including loop)} that you will use to write your program.
The details of subalgorithms are only required for procedures that are new for this project, not those provided to you.
Instead, just mention what procedure will be used and the inputs and outputs of the procedure, e.g. ``Calculate orbital elements given position and velocity vectors using \texttt{rv2coe}''.

When completed, anyone should be able to write your code \textbf{solely} using the algorithm in \textbf{any} computer language of their choice.
Thus, define all symbols before you use them, and do not write the equations and logic using any language specific terminology, i.e. Something like `` Find the length of the vector using \texttt{norm(x)} '' is unacceptable.

Your algorithm must be \textbf{typed}, which will serve you well when you document your final code. 
This is also a good opportunity to practice your technical writing skills in \LaTeX.

\subsection*{Final COMFIX Deliverables}

Your program must process the \texttt{COMFIX.DAT} data file and generate output that matches the orbital elements in \texttt{COMFIX.OUT} to at least four decimal places. 
The input file, \texttt{COMFIX.DAT}, contains five observations, but your program should be able to process an arbitrary number of observations.

Submit the following on the due date:
\begin{itemize}
    \item Fully documented driver script
    \item Each procedure which you wrote or modified, fully documented (no library routines)
    \item Computer generated results which match \texttt{COMFIX.OUT}
\end{itemize}

\subsection*{Software Library}

A library of support functions are provided for your use. 
This library gives you a model for the structure of your project as well as several functions you can use in the development of your own programs.
Furthermore, the library shows the standards used in this class for:
\begin{itemize}
    \item Proper documentation
    \item Proper testing
\end{itemize}

\subsection*{Program Specifications}

The following is a description of the required functions that your software program must include. 
You will be making use of many of the functions that you have been developing over the course of the semester.

\subsubsection*{\texttt{gstlst}}

This function computes the Greenwich sidereal time and Local sidereal time at a specific site given a known Julian date.

\noindent \textbf{Inputs: }
\begin{itemize}
    \item \( jd \) - Julian date
    \item \( \lambda \) - site longitude in radians
\end{itemize}

\noindent \textbf{Output:}
\begin{itemize}
    \item \(gst \) - Greenwich sidereal time in radians
    \item \(lst\) - Local sidereal time in radians
\end{itemize}

This function will make use of \texttt{gsttime0} which provides the Greenwich sidereal time at the beginning of the year. 

\subsubsection*{\texttt{site}}
This function computes the Earth Centered Inertial position vectors of an observation site on the Earth.

\noindent \textbf{Inputs: }
\begin{itemize}
    \item \( \phi \) - site geodetic latitude in radians
    \item \( \lambda \) - site longitude in radians
    \item \( h \) - site altitude above mean sea level in kilometers
\end{itemize}

\noindent \textbf{Output:}
\begin{itemize}
    \item \( r_{eci} \in \R^{3 \times 1} \) - site position vector in the Earth centered inertial frame and given in kilometers
\end{itemize}

\subsubsection*{\texttt{rvtopos}}
This function converts radar measurements to a range and range rate vector in the topocentric reference frame.

\noindent \textbf{Input : }
\begin{itemize}
    \item \( \rho \) - range from the site to spacecraft in kilometers
    \item \( \alpha \) - azimuth in topocentric reference frame in radians
    \item \( \beta \) - elevation in topocentric reference frame in radians 
    \item \( \dot{\rho} \) - range rate in kilometers per second
    \item \( \dot{\alpha} \) - azimuth rate in radians per second
    \item \( \dot{\beta} \) - elevation rate in radians per second
\end{itemize}

\noindent \textbf{Output: }

\begin{itemize}
    \item \( r_{sez} \) - position of spacecraft in topocentric reference frame in kilometers
    \item \( v_{sez} \) - velocity of spacecraft in the topocentric reference frame in kilometers per second
\end{itemize}
\end{document}  
