\documentclass[10pt]{article}
\usepackage[letterpaper,text={6.5in,8.7in},centering]{geometry}
\usepackage{amssymb,amsmath,times,url,subfigure,graphicx,multirow}
\usepackage[pdftex,urlcolor=blue,pdfpagemode=none,pdfstartview=FitH]{hyperref}
\usepackage[final]{pdfpages}
\usepackage{biblatex}
\addbibresource{library.bib}
%% url smaller font.
\makeatletter
\def\url@leostyle{%
  \@ifundefined{selectfont}{\def\UrlFont{\sf}}{\def\UrlFont{\small\ttfamily}}}
\makeatother
\urlstyle{leo}

%\usepackage[all,import]{xy}


\renewcommand{\baselinestretch}{1.2}
\date{}

\renewcommand{\thesubsection}{\arabic{subsection}. }
\renewcommand{\thesubsubsection}{\arabic{subsection}.\arabic{subsubsection} }


\begin{document}
\pagestyle{empty}
\section*{MAE 3145: Orbital Mechanics \& Space Dynamics}
\vspace*{-0.4cm}
\noindent{Fall 2018, W 1810--2040, SEH 3040}

%From 2012, Delete Chapter 5, and revise the course calendar.

\paragraph*{Instructor}
Shankar Kulumani\quad Email: \href{mailto:skulumani@gwu.edu}{skulumani@gwu.edu}\quad Homepage: \url{http://fdcl.seas.gwu.edu/}\\

\paragraph*{Prerequisites} APSC 2058 Analytical Mechanics II

\paragraph*{Course Description} This course covers the motion of spacecraft under gravity. Included are the derivation and the analyses of the two-body problem and their applications for real world missions.
Extensive use of scientific programming languages is required to simulate orbital dynamics and solve real-world astrodynamics problems.

\paragraph*{Textbook}
\begin{list}{$\bullet$}{\setlength{\itemsep}{-3pt}}
\item R. Bate, \textit{Fundamentals of Astrodynamics}, Dover Publication, 1971
\end{list}


\paragraph*{Contents}
\begin{list}{$\bullet$}{\setlength{\itemsep}{-3pt}}
\item Astrodynamic Fundamentals
    \begin{list}{$-$}{\setlength{\itemsep}{-3pt}}
    \item Time [BMW Chap. 2.9]
    \item Coordinate Systems [ BMW Chap. 7.4]
    \item Project RV2COE [BMW Chap 2.5 ]
    \end{list}
\item Orbital Mechanics\vspace*{-0.2cm}
    \begin{list}{$-$}{\setlength{\itemsep}{-3pt}}
    \item Dynamics of Point Masses [Curtis Chap. 1]
    \item Two-Body Problem [ BMW Chap. 1]
    \item Orbital Elements [ BMW Chap. 2.3]
    \item Groundtracks [ BMW Chap. 2.15]
    \item Project COMFIX [BMW Chap 2.7 ]
    \item Project PROPOGATE [ BMW Chap 4 ]
    \end{list}
%\item Orbit Determination\vspace*{-0.2cm}
%    \begin{list}{$-$}{\setlength{\itemsep}{-3pt}}
%    \item Gibbs Method [Chap. 5]
%    \end{list}
\item Orbital Maneuvers\vspace*{-0.2cm}
    \begin{list}{$-$}{\setlength{\itemsep}{-3pt}}
    \item Hohmann Transfers [ BMW Chap. 3]
    \item Plane Changes [ BMW Chap. 3.4]
    \item Orbital Rendezvous and Phasing [ BMW Chap. 8.3]
    \end{list}
\item Perturbations [BMW Chap. 8]\vspace*{-0.2cm}
    \begin{list}{$-$}{\setlength{\itemsep}{-3pt}}
    \item Geopotential 
    \item Drag
    \item PREDICT 
    \end{list}
\item Extras if we're lucky\vspace*{-0.2cm}
    \begin{list}{$-$}{\setlength{\itemsep}{-3pt}}
        \item Method of patched conics
        \item Gauss Problem
        \item Lambert Problem
    \end{list}
\end{list}

\paragraph*{Software Projects}
A major focus of this course will be the application of sound scientific programming skills.
You will apply the theoretical tools of astrodynamics to solve realistic problems by implementing your own library of tools.

\begin{list}{$\bullet$}{\setlength{\itemsep}{-3pt}}
    \item RV2COE - convert position and velocity vectors of a spacecraft to classical orbital elements
    \item COMFIX - determine the orbital elements of a satellite given ground based radar observations
    \item PROPOGATE - determine the position of a satellite as a function of time
    \item PREDICT - predict satellite passes for any location on the Earth
    \end{list}
\paragraph*{Additional Readings - Other sources of useful information}
\begin{list}
{$\bullet$}
{\setlength{\itemsep}{-3pt}}
\item \fullcite{vallado2007}
\item \fullcite{battin1999}
\item J. Danby, \textit{Fundamentals of Celestial Mechanics}, Willmann-Bell, 1988
\item J. Prussing, \textit{Orbital Mechanics}, Oxford University Press, 1993
\item V. Chobotov, \textit{Orbital Mechanics}, AIAA, 2002
\item T. Logsdon, \textit{Orbital Mechanics: Theory and Applications}, Wiley, 1997
%%\item R. Dorf, and R. Bishop, \textit{Modern Control Systems}, Prentice Hall, 2007
%\item K. Ogata, \textit{Modern Control Engineering}, Prentice Hall, 2001
%\item N. Leonard and W. Levine, \textit{Using MATLAB to Analyze and Design Control Systems}, Addison Wesley, 1995
\item Scipy Tutorial: \href{http://www.scipy-lectures.org/}{http://www.scipy-lectures.org/}
\end{list}
%\renewcommand{\thesubfigure}{}
%\begin{figure}[h]\vspace*{-0.3cm}
%\centerline{
%\subfigure[Etkin]{
%    \href{http://books.google.com/books?id=AIIOAAAACAAJ}%
%    {\includegraphics[height=2.3cm]{etkin.pdf}}}
%\subfigure[Nelson]{
%    \href{http://books.google.com/books?id=bt46AAAACAAJ}%
%    {\includegraphics[height=2.3cm]{nelson.jpg}}}
%\hspace*{0.5cm}
%\subfigure[Stevens]{
%    \href{http://books.google.com/books?id=T0Ux6av4btIC}%
%    {\includegraphics[height=2.3cm]{stevens.jpg}}}
%\subfigure[Stengel]{
%    \href{http://books.google.com/books?id=nI_EHQAACAAJ&dq=stengel+flight&ei=YpKkSL7tL4L2iQG7hZ36BA}%
%    {\includegraphics[height=2.3cm]{stengel.jpg}}}
%\subfigure[Roskam]{
%    \href{https://oscommerce.darcorp.com/product_info.php?cPath=21&products_id=125&osCsid=077f051b037aa4a35b9b6e52998bbae8}%
%    {\includegraphics[height=2.3cm]{roskam.jpg}}}
%\hspace*{0.5cm}
%\subfigure[Anderson]{
%    \href{http://books.google.com/books?id=Hd_AR0CAmsoC}%
%    {\includegraphics[height=2.3cm]{anderson.jpg}}}
%\subfigure[McClamroch]{
%    \href{http://my.fit.edu/~taeyoung/McClamroch.pdf}%
%    {\includegraphics[height=2.3cm]{nhm.pdf}}}}
%\end{figure}

\paragraph*{Grading}
Homework 35\%,\;\; Attendance 5\%,\;\; Midterm Exam 20\% ,\;\; Final Exam 20\%, \;\; Projects 20\%


\paragraph*{Course Learning Objectives}
At the end of this course, students will be able to:

\newcounter{lcounter}
\begin{list}
{\arabic{lcounter}:}
{\setlength{\itemsep}{-3pt}\usecounter{lcounter}}
\item Explain the Newtonian gravitational force and gravitational potential between particles
%\item Understand the dynamics of a system of two particles acting under their mutual gravitational potential
\item Analyze the characteristics of circular, elliptic, parabolic and hyperbolic orbits in a two-dimensional plane
\item Describe the geometry of an orbit in a three-dimensional space from orbital elements
\item Apply numerical/analytical techniques to propogate orbits.
\item Choose and apply the approprite orbital maneuvering method to move spacecraft between orbits.
\item Develop personal software tools to solve practical astrodynamic problems:
    \begin{itemize}
        \item Determine orbital parameters from ground based observations.
        \item Predict satellite passes and determine observation angles to view satellites overhead.
    \end{itemize}
\end{list}


\paragraph*{ Minimum out of class work per week} Expect to spend much more time than this.
\begin{itemize}
    \item Direct Instruction - 2.5 hours
    \item Indendent Learning - 5 hours
\end{itemize}

\newpage
\paragraph*{General Policy}
\begin{list}
{\arabic{lcounter}:}
{\setlength{\itemsep}{-3pt}\usecounter{lcounter}}


\item If you do not have a computer account for accessing Python, you should contact SEAS Computing Facility

\item Class attendance is required. Be punctual and follow the class rules. Students are encouraged to ask questions but talking while the lecture is being delivered is prohibited.


\item If there are any questions regarding grading, you must attach a written explanation of your issue to your assigment and return both to the instructor within one week.

\item Class/Lab cancellation due to weather or special event: Call 202-994-5050 or visit the Campus Advisories Website (\url{campusadvisories.gwu.edu}) or \url{www.gwu.edu/~bygeorge/100703/closingpolicy.html} for GW operating status. 

\item Disability Support Services (DSS): If a student is to use DSS for testing, he/she should submit the letter from DSS during the first week. Any student who may need an accommodation based on the potential impact of a disability should contact the Disability Support Services office (\url{http://gwired.gwu.edu/dss/}) during the first week.

\item Students requiring special accommodations for testing through DSS must provide Dr. Kulumani with the appropriate forms or documents and confirm the approval at least two weeks before the test or exam.
\end{list}


\paragraph*{Academic Integrity Code}

Academic dishonesty is defined as cheating of any kind, including misrepresenting one's own work, taking credit for the work of others without crediting them and without appropriate authorization, and the fabrication of information. 

All academic integrity violations will be reported in accordance with the GWU code of academic integrity, see: \url{http://studentconduct.gwu.edu/code-academic-integrity}.


\paragraph*{Email Policy}

\begin{list}
{$\bullet$}
{\setlength{\itemsep}{-3pt}}

\item \textbf{Check your GW email account daily}. All of the important announcements of this class will be made through Blackboard/email. 

\item I will not respond to emails which are composed in an unprofessional manner, or which violates basic email etiquette. 

\item I do not offer immediate round the clock technical support, please plan ahead accordingly.
I will try to respond to emails within 36 hours during the week, and within 72 hours during the weekend.

\end{list}

\paragraph*{Homework Policy}

\begin{list}
{$\bullet$}
{\setlength{\itemsep}{-3pt}}

\item Homework is due at the \textbf{beginning of class}.

\item \textbf{Late homework} assignments \textbf{WILL NOT} be accepted \textbf{for any reason}.

\item Grading of the homework will emphasize your effort to present the solution in a neat and orderly fashion.

% \begin{list}
% {$-$}
% {\setlength{\itemsep}{-3pt}}
% \item Use one side of a clean paper (graphed or lined is okay) that is not torn from a spiral notebook.
% \item Write your name, ID number, and section clearly on the front page of your completed assignment.
% \item Clearly number each solution and present them in numerical order.
% \item Leave at least one line of space between each problem.
% \item Write clearly and legibly.
% \item Use a stapler.
% \end{list}

\end{list}

\paragraph*{Exam Policy}

\begin{list}
{$\bullet$}
{\setlength{\itemsep}{-3pt}}


\item There is one midterm exam and one final exam. 
\item \textbf{Make-up exams will not be given }. 
    If there are exceptional situations you may discuss with the instructor.

\end{list}

\end{document}

