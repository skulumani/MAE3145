\documentclass[10pt]{article}
% \usepackage[letterpaper,text={6.5in,8.7in},centering]{geometry}
\usepackage{curves}
\usepackage{epic,eepic,color}
%\usepackage[usenames,dvipsnames,svgnames,table]{xcolor}
% \usepackage{amssymb,amsmath,times,subfigure,graphicx,theorem}
\usepackage{alltt}
%\usepackage{warmread}
%\usepackage[all,import]{xy}
%\usepackage{eepic}
\usepackage{my_packages}
\usepackage{lastpage}
\usepackage{fancyhdr}
\pagestyle{fancy}
\cfoot{\thepage\ of \pageref{LastPage}}
\renewcommand{\headrulewidth}{0pt}
\usepackage{tikz_packages}
\renewcommand{\baselinestretch}{1.2}
\date{}

\renewcommand{\thesubsection}{\arabic{subsection}. }
\renewcommand{\thesubsubsection}{\arabic{subsection}.\arabic{subsubsection} }

\theoremstyle{definition}
\newtheorem{prob}{Problem}[section]
%\renewcommand{\theprob}{\arabic{section}.\arabic{prob}}
\renewcommand{\theprob}{\arabic{prob}}

\newenvironment{subprob}%
{\renewcommand{\theenumi}{\alph{enumi}}\renewcommand{\labelenumi}{(\theenumi)}\begin{enumerate}}%
{\end{enumerate}}%


%1: Explain the Newtonian gravitational force and gravitational potential between particles
%2: Analyze the characteristics of circular, elliptic, parabolic and hyperbolic orbits in a two-dimensional plane
%3: Apply numerical/analytical techniques to propogate orbits
%4: Describe the geometry of an orbit in a three-dimensional space from orbital elements


\begin{document}


\setcounter{page}{0}

\vspace*{1cm}

\centerline{\LARGE{ MAE3145: Midterm Exam}}
\vspace*{0.5cm}
% \centerline{\Large \SI{2458416.465}{\julianday}}%\\%\vspace*{0.5cm}

\vspace*{6cm}

\centerline{
\begin{tabular}{lll}
\hspace*{5cm}, & \hspace*{5cm}. & \hspace*{4cm}\\\hline
Last Name & First Name & Student ID
\end{tabular}}

\vspace*{6cm}

% \centerline{
% \begin{tabular}{|c|c|c|c|c|c|}\hline
%     Prob. 1 & Prob. 2 & Prob. 3 & Prob. 4 & Prob. 5 & Total \\
%     (20) & (20) & (20) & (20) & (20) & (100)\\ \hline
%     \hspace*{2.2cm} & \hspace*{2.2cm} & \hspace*{2.2cm} & \hspace*{2.2cm} & \hspace*{2.2cm}  & \hspace*{2.2cm} \\
%                     & & & & & \\
%                 & & & & & \\
% \hline
% \end{tabular}}

\clearpage\newpage
% \renewcommand{\theprob}{\arabic{prob} \textit{(15pt)}}



\begin{prob}
    \textbf{20pt}
    Due to your extensive orbital mechanics training at the George Washington University, you have been hand selected to lead the Washington DC small body defense system. 
    The goal of the small body defense system is to intercept incoming bodies before they strike our nations capital. 
    On your first day at work a large body is on course to strike the District. 
    At your disposal is a single ground launched missile which you can use to intercept and destroy the body. 
    Use your skills in astrodynamics to save the District. 

    Ground based measurements provided the following orbital properties for the incoming body:
    \begin{align}
        r_p = \SI{6000}{\kilo\meter}, \quad e = 1.2.
    \end{align}
    Your launch system can deliver a missile onto an orbit defined by the following:
    \begin{align}
        r_p = \SI{2000}{\kilo\meter}, \quad r_a = \SI{6500}{\kilo\meter} .
    \end{align}

\begin{subprob}
    \item Currently, the incoming body is at an altitude of \SI{36000}{\kilo\meter}.
        In order to plan the evacuation, in case your calculations are incorrect, the District needs to determine the amount of time remaining before impact.
        Determine the amount of time for the incoming body to travel from its current altitude to the surface.
    \item In order to intercept the body, you need to determine the time of flight of the missile from the launch site to apogee. 
        You can assume that the launch site is located at sea level, i.e. the radius of the launch site is \( r_l = \SI{6378.137}{\kilo\meter}\).
    \item Assuming you can intercept the body at apogee, \( r = \SI{6500}{\kilo\meter} \), what would be the magnitude of the closing velocity at impact.
        In other words, determine the velocity of the missile and body as the intercept radius.
\end{subprob}
\end{prob}

\clearpage\newpage
\null\newpage
\null\newpage
\begin{prob}
    \textbf{20pt}
    An orbital vehicle about the Earth has the following classical orbital elements:
    \begin{align}
        p = 0.23 R_{E}, \quad e = 0.82, \quad i = \SI{90}{\degree}, \\
        \Omega = \SI{180}{\degree}, \quad \omega = \SI{260}{\degree}, \quad \nu = \SI{190}{\degree}.
    \end{align}
   
\begin{subprob}
    \item Determine the position and velocity vector in the inertial frame.
    \item Determine the following orbital properties of the vehicle.
        \begin{itemize}
            \item Radius of periapsis
            \item Radius of apoapsis
            \item Specific Mechanical Energy
            \item Flight Path angle
            \item Eccentric anomaly
            \item Specific Mechanical Energy
        \end{itemize}
\end{subprob}
\end{prob}
\clearpage\newpage
\null\newpage
\null\newpage


\end{document}
