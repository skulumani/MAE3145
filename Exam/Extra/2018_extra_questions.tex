\documentclass[10pt]{article}
% \usepackage[letterpaper,text={6.5in,8.7in},centering]{geometry}
\usepackage{curves}
\usepackage{epic,eepic,color}
%\usepackage[usenames,dvipsnames,svgnames,table]{xcolor}
% \usepackage{amssymb,amsmath,times,subfigure,graphicx,theorem}
\usepackage{alltt}
%\usepackage{warmread}
%\usepackage[all,import]{xy}
%\usepackage{eepic}
\usepackage{my_packages}
\usepackage{lastpage}
\usepackage{fancyhdr}
\pagestyle{fancy}
\cfoot{\thepage\ of \pageref{LastPage}}
\renewcommand{\headrulewidth}{0pt}
\usepackage{tikz_packages}
\renewcommand{\baselinestretch}{1.2}
\date{}

\renewcommand{\thesubsection}{\arabic{subsection}. }
\renewcommand{\thesubsubsection}{\arabic{subsection}.\arabic{subsubsection} }

\theoremstyle{definition}
\newtheorem{prob}{Problem}[section]
%\renewcommand{\theprob}{\arabic{section}.\arabic{prob}}
\renewcommand{\theprob}{\arabic{prob}}

\newenvironment{subprob}%
{\renewcommand{\theenumi}{\alph{enumi}}\renewcommand{\labelenumi}{(\theenumi)}\begin{enumerate}}%
{\end{enumerate}}%


%1: Explain the Newtonian gravitational force and gravitational potential between particles
%2: Analyze the characteristics of circular, elliptic, parabolic and hyperbolic orbits in a two-dimensional plane
%3: Apply numerical/analytical techniques to propogate orbits
%4: Describe the geometry of an orbit in a three-dimensional space from orbital elements


\begin{document}


\setcounter{page}{0}

\vspace*{1cm}

\centerline{\LARGE{ MAE3145: Midterm Exam}}
\vspace*{0.5cm}
\centerline{\Large \SI{2458416.465}{\julianday}}%\\%\vspace*{0.5cm}

\vspace*{6cm}

\centerline{
\begin{tabular}{lll}
\hspace*{5cm}, & \hspace*{5cm}. & \hspace*{4cm}\\\hline
Last Name & First Name & Student ID
\end{tabular}}

\vspace*{6cm}

\centerline{
\begin{tabular}{|c|c|c|c|c|c|}\hline
    Prob. 1 & Prob. 2 & Prob. 3 & Prob. 4 & Prob. 5 & Total \\
    (20) & (20) & (20) & (20) & (20) & (100)\\ \hline
    \hspace*{2.2cm} & \hspace*{2.2cm} & \hspace*{2.2cm} & \hspace*{2.2cm} & \hspace*{2.2cm}  & \hspace*{2.2cm} \\
                    & & & & & \\
                & & & & & \\
\hline
\end{tabular}}

\clearpage\newpage
% \renewcommand{\theprob}{\arabic{prob} \textit{(15pt)}}



\begin{prob}
    \textbf{20pt}
Consider an elliptical  orbit.
Define \( t_{outer} \) as the time required to move from a point on one end of the minor axis, through apoapsis, to a point on the other end of the minor axis.

\begin{subprob}
    \item Write an expression for the ratio of \( t_{outer} \) to the orbital period, i.e. \(\frac{t_{outer}}{\mathbb{P}}\).
    \item If \( e = \frac{3}{4} \), the time spent in the outer half of the orbit is what percentage of the total period? In other words, find the ratio \( \frac{t_{outer}}{\mathbb{P}} \).
\end{subprob}
\end{prob}

\clearpage\newpage
\null\newpage
\null\newpage
\begin{prob}
    \textbf{20pt}
    Assume that a spacecraft is in the orbit about some planet of radius \( R\) and it is reasonable to model the orbit in terms of the two-body problem.
    The perifocal set of unit vectors are \( \hat p \) and \( \hat q \).

    At a given instant, the spacecraft is located at the end of the minor axis such that:
    \begin{align*}
        \bar r &= -4 R \hat p - 4 \sqrt{3} R \hat q \\
        \norm{\bar v} &= \SI{3}{\radian\per\second}
    \end{align*}

\begin{subprob}
\item Determine the following, where \( a \) is the semimajor axis, \( b \) is the semiminor axis, \( p \) is the semilatus rectum, \( e \) is eccentricity, \( \gamma \) is the flight path angle, \( \mathcal{E} \) is the specific mechanical energy, \( E \) is eccentric anomaly, and \( h \) is the specific angular momentum.
        \begin{align*}
            \frac{a}{R}, \quad \frac{b}{R}, \quad \frac{p}{R}, \quad e, \quad \gamma, \quad \nu, \quad E, \quad \mathcal{E}, \quad \frac{h}{R}
        \end{align*}
    \item Sketch the orbit and mark \( \bar r, \bar v , \gamma, \nu, E\) and the local horizontal and local vertical frame.
\end{subprob}
\end{prob}
\clearpage\newpage
\null\newpage
\null\newpage


\end{document}
