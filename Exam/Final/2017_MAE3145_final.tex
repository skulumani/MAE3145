\documentclass[10pt]{article}
% \usepackage[letterpaper,text={6.5in,8.7in},centering]{geometry}
\usepackage{curves}
\usepackage{epic,eepic,color}
%\usepackage[usenames,dvipsnames,svgnames,table]{xcolor}
% \usepackage{amssymb,amsmath,times,subfigure,graphicx,theorem}
\usepackage{alltt}
%\usepackage{warmread}
%\usepackage[all,import]{xy}
%\usepackage{eepic}
\usepackage{my_packages}
\usepackage{lastpage}
\usepackage{fancyhdr}
\pagestyle{fancy}
\cfoot{\thepage\ of \pageref{LastPage}}
\renewcommand{\headrulewidth}{0pt}
\usepackage{tikz_packages}
\renewcommand{\baselinestretch}{1.2}
\date{}

\renewcommand{\thesubsection}{\arabic{subsection}. }
\renewcommand{\thesubsubsection}{\arabic{subsection}.\arabic{subsubsection} }

\theoremstyle{definition}
\newtheorem{prob}{Problem}[section]
%\renewcommand{\theprob}{\arabic{section}.\arabic{prob}}
\renewcommand{\theprob}{\arabic{prob}}

\newenvironment{subprob}%
{\renewcommand{\theenumi}{\alph{enumi}}\renewcommand{\labelenumi}{(\theenumi)}\begin{enumerate}}%
{\end{enumerate}}%


%1: Explain the Newtonian gravitational force and gravitational potential between particles
%2: Analyze the characteristics of circular, elliptic, parabolic and hyperbolic orbits in a two-dimensional plane
%3: Apply numerical/analytical techniques to propogate orbits
%4: Describe the geometry of an orbit in a three-dimensional space from orbital elements


\begin{document}


\setcounter{page}{0}

\vspace*{1cm}

\centerline{\LARGE{ MAE3145: Final Exam}}
\vspace*{0.5cm}
\centerline{\Large \SI{2458052.1979}{\julianday}}%\\%\vspace*{0.5cm}

\vspace*{6cm}

\centerline{
\begin{tabular}{lll}
\hspace*{5cm}, & \hspace*{5cm}. & \hspace*{4cm}\\\hline
Last Name & First Name & Student ID
\end{tabular}}

\vspace*{6cm}

\centerline{
\begin{tabular}{|c|c|c|c|c|}\hline
Prob. 1 & Prob. 2 & Prob. 3 & Prob. 4 & Total \\
(18) & (40) & (22) & (20) & (100)\\ \hline
\hspace*{2.2cm} & \hspace*{2.2cm} & \hspace*{2.2cm} & \hspace*{2.2cm} & \hspace*{2.2cm}\\
                & & & & \\
                & & & & \\
\hline
\end{tabular}}

\clearpage\newpage
% \renewcommand{\theprob}{\arabic{prob} \textit{(15pt)}}

\begin{prob}
    Your spacecraft is currenlty in a circular orbit about Planet X with \( \Omega = \SI{90}{\degree}\) and \( i = \SI{30}{\degree} \) relative to an inertial reference frame defined by \( \hat{x}, \hat{y}, \hat{z} \).
    At the descending node, the following maneuver is implemented:
    \begin{align*}
        \bar \Delta v = \frac{1}{\sqrt{2}} \hat{V} - \hat{C} + \sqrt{\frac{3}{2}} \hat{N} \si{\kilo\meter\per\second} .
    \end{align*}

    Express the \( \Delta \bar v \) in terms of :
    \begin{subprob}
        \item Inertial reference frame: \( \hat x, \hat y, \hat z\)
        \item \(\norm{\Delta \bar v}, \alpha, \beta\) relative to the \( \hat V, \hat N, \hat C\) reference frame
    \end{subprob}
\end{prob}
\clearpage\newpage
\null\newpage
\null\newpage
\end{document}
