\documentclass[10pt]{article}
% \usepackage[letterpaper,text={6.5in,8.7in},centering]{geometry}
\usepackage{curves}
\usepackage{epic,eepic,color}
%\usepackage[usenames,dvipsnames,svgnames,table]{xcolor}
% \usepackage{amssymb,amsmath,times,subfigure,graphicx,theorem}
\usepackage{alltt}
%\usepackage{warmread}
%\usepackage[all,import]{xy}
%\usepackage{eepic}
\usepackage{my_packages}
\usepackage{lastpage}
\usepackage{fancyhdr}
\pagestyle{fancy}
\cfoot{\thepage\ of \pageref{LastPage}}
\renewcommand{\headrulewidth}{0pt}
\usepackage{tikz_packages}
\renewcommand{\baselinestretch}{1.2}
\date{}

\renewcommand{\thesubsection}{\arabic{subsection}. }
\renewcommand{\thesubsubsection}{\arabic{subsection}.\arabic{subsubsection} }

\theoremstyle{definition}
\newtheorem{prob}{Problem}[section]
%\renewcommand{\theprob}{\arabic{section}.\arabic{prob}}
\renewcommand{\theprob}{\arabic{prob}}

\newenvironment{subprob}%
{\renewcommand{\theenumi}{\alph{enumi}}\renewcommand{\labelenumi}{(\theenumi)}\begin{enumerate}}%
{\end{enumerate}}%


%1: Explain the Newtonian gravitational force and gravitational potential between particles
%2: Analyze the characteristics of circular, elliptic, parabolic and hyperbolic orbits in a two-dimensional plane
%3: Apply numerical/analytical techniques to propogate orbits
%4: Describe the geometry of an orbit in a three-dimensional space from orbital elements


\begin{document}


\setcounter{page}{0}

\vspace*{1cm}

\centerline{\LARGE{ MAE3145: Final Exam}}
\vspace*{0.5cm}
\centerline{\Large \SI{2458052.1979}{\julianday}}%\\%\vspace*{0.5cm}

\vspace*{6cm}

\centerline{
\begin{tabular}{lll}
\hspace*{5cm}, & \hspace*{5cm}. & \hspace*{4cm}\\\hline
Last Name & First Name & Student ID
\end{tabular}}

\vspace*{6cm}

\centerline{
\begin{tabular}{|c|c|c|c|c|}\hline
Prob. 1 & Prob. 2 & Prob. 3 & Prob. 4 & Total \\
(18) & (40) & (22) & (20) & (100)\\ \hline
\hspace*{2.2cm} & \hspace*{2.2cm} & \hspace*{2.2cm} & \hspace*{2.2cm} & \hspace*{2.2cm}\\
                & & & & \\
                & & & & \\
\hline
\end{tabular}}

\clearpage\newpage
% \renewcommand{\theprob}{\arabic{prob} \textit{(15pt)}}


\begin{prob}
    A vehicle is in a circular orbit about the Earth with radius \( r_1 = 6 R_\oplus\).
    A Hohmann transfer is employed to shift to a smaller, coplanar  circular orbit of \( r_2 = 2 R_\oplus\).
    \begin{subprob}
        \item Determine \( \Delta \bar v_{\text{total}}\) and the time of flight.
        \item Indicate if each manuever increase or decreases the instaneous speed.
    \end{subprob}
\end{prob}


\begin{prob}
    Develop an algorithm to determine the \textbf{PERIOD OF THE PHASING ORBIT} for a non-coplanar rendezvous problem to deploy a satellite from an inclined circular low Earth orbit to a higher, circular equatorial orbit at the \textbf{first opportunity}.
    The first  few steps have been outlined for you; complete the remaining algorithm.

\noindent\textbf{GIVEN:} \\
        Interceptor satellite COEs: \( a_{int}, i, \Omega, \theta\)\\
        Target satellite COEs: \( a_{tgt}, \theta\)

\noindent\textbf{FIND:} \\
Period of phasing orbit : \( \mathbb{P}_{\text{phasing}}\)

\begin{enumerate}
\item Calculate the angular speed ( mean motion) for both interceptor and target.
    \begin{align*}
        \omega_{int} = \sqrt{\frac{\mu}{a_{int}^3}} \qquad \omega_{tgt} = \sqrt{\frac{\mu}{a_{tgt}^3}}
    \end{align*}

\item Calculate the TOF for the Hohmann transfer (complete the equation below).
    \vspace*{2cm}
\item Calculate ...
\end{enumerate}
\end{prob}

\begin{prob}
    A radar tracking site is located at the following location: (assume a perfectly spherical Earth)
    \begin{itemize}
        \item Latitude: \SI{90}{\degree} North
        \item Altitude: \SI{0}{\kilo\meter}
        \item Local Sidereal Time: \SI{180}{\degree}
    \end{itemize}

    A satellite is in a circular polar orbit with \( a = \SI{9020.5}{\kilo\meter}, \Omega = \SI{90}{\degree}, \theta = \SI{45}{\degree}\).

    Determine the following:
    \begin{subprob}
    \item \textbf{Range-Vector} from the site to the satellite in the Earth Centered Inertial Reference frame.
    \item \textbf{Elevation angle} and \textbf{Range} from the site to the satellite
    \end{subprob}
\end{prob}
\end{document}
